\chapter{Indledning} 
Kronisk obstruktiv lungesygdom (KOL) er en inflammatorisk sygdom, der ødelægger bronkiernes vægge og/eller danner forsnævringer i luftvejene. Dette forårsager, at lungefunktionen gradvist nedsættes.\cite{Basisbogen2016} Bronkiernes ødelagte vægge reducerer lungernes overflade, som mindsker luftudvekslingen. Forsnævringerne i luftvejene blokerer, hvorved luft ikke længere kan passere frit igennem. Det kræver derfor mere arbejde ved ventilation end normalt.\cite{Lungeforeningen2016}

I Danmark er der ca. 430.000 mennesker med KOL, hvortil der årligt er 10.000 nye tilfælde \cite{Sygdomsbyrden2015}. Den årlige mortalitet er 3.500, hvilket gør KOL til den fjerde hyppigste dødsårsag i Danmark.\cite{Basisbogen2016} På verdensplan er KOL på nuværende tidspunkt den tredje hyppigste dødsårsag \cite{WHO2017}.

KOL opstår af skadelige partikler samt gasser og miljøpåvirkninger. Tobaksrygning samt passiv rygningen udgør $85-90~\%$ af tilfældene, hvilket gør disse til de hyppigste årsager til KOL.\cite{Basisbogen2016,Sygdomsbyrden2015,dsam2016,Martinez2016}
Miljøpåvirkninger kan blandt andet være dårligt arbejdsmiljø eller opvækst i dårligt miljø. Dårligt arbejdsmiljø er eksempelvis arbejde med asbest, hvorimod dårligt miljø kan medvirke til, at barnets lunger ikke udvikler sig ordentligt. Disse faktorer kan derved resultere i en accelererende reduktion af lungefunktionen.\cite{Martinez2016}

Lungefunktionen nedsættes gradvist over mange år, hvilket gør, at KOL først kommer til udtryk sent i sygdomsforløbet. Dette kan resultere i, at patienter først opsøger læge, når lungefunktionen er halveret.\cite{dsam2016} Symptomer forbundet med KOL opleves som åndenød samt hoste ved fysisk aktivitet, derudover er der en tendens til hyppige eksacerbationer. Eksacerbationer er akut forværring af patienters tilstand, hvilket kræver behandling.\cite{Basisbogen2016,dsam2016}
KOL-patienter oplever åndenød, hvilket kan medføre svage perifere muskler. Dette kan lede til en række komorbiditeter. Disse fremtræder som kardiovaskulære sygdomme, type-2 diabetes, osteoporose, lungecancer og muskelsvækkelse. Foruden de nævnte komorbiditeter, kan patienterne ligeledes opleve psykiske komorbiditeter, såsom depression og angst, da patienterne ofte isolerer sig på grund af generne ved KOL.\cite{dsam2016}

KOL kan ikke helbredes, og det er dertil ikke muligt at genvinde den tabte lungefunktion. Dog er det muligt at forhindre yderligere tab af lungefunktionen forårsaget af KOL samt lindre patienters symptomer.\cite{Basisbogen2016} Dette leder op til følgende initierende problemstilling.


\section{Initierende problemstilling}
\textit{Hvordan er nuværende diagnosticering og behandling af patienter med kronisk obstruktiv lungesygdom og, hvilke rehabiliteringsmuligheder kan tilbydes?}