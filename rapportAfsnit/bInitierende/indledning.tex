\chapter{Indledning} 
Kronisk obstruktiv lungesygdom (KOL) er en kronisk inflammatorisk lungesygdom, der ødelægger bronkiernes vægge og/ellers danner forsnævringer i luftvejene. Dette forårsager, at lungefunktionen gradvist nedsættes.\cite{Basisbogen2016}

I Danmark er der ca. 430.000 mennesker med KOL, hvortil der er en årlig mortalitet på 3.500. Dette gør KOL til den fjerde hyppigste dødsårsag i Danmark.\cite{Basisbogen2016} På verdensplan er KOL på nuværende tidspunkt den tredje hyppigste dødsårsag \cite{WHO2017}.

KOL opstår som ofte af skadelige partikler samt gasser og miljøpåvirkninger. Den hyppigste årsag til KOL er tobaksrygning, der fremskynder tabet af lungefunktionen.\cite{dsam2016,Basisbogen2016,Martinez2016} Miljøpåvirkninger kan blandt andet være dårligt arbejdsmiljø, som eksempelvis arbejde med asbest, eller opvækst i dårligt miljø, hvilket kan påvirke barnets lunger til ikke at udvikle sig ordentligt. Miljøpåvirkninger kan derved resultere i en accelererende reduktion i lungefunktionen.\cite{Martinez2016}

Lungefunktionen nedsættes gradvist over mange år, hvilket gør, at KOL først kommer til udtryk sent i sygeforløbet. Dette kan resultere i, at patienter først opsøger sin læge, når deres lungefunktion er halveret.\cite{dsam2016} Symptomer forbundet med KOL opleves som åndenød samt hoste ved fysisk aktivitet, derudover er der en tendens til hyppig eksacerbationer. Eksacerbationer er akut forværring af patientens tilstand, hvilket kræver behandling.\cite{dsam2016,Basisbogen2016}
Derudover er der en række komorbiditeter, der kan være forårsaget af åndenød samt svage perifere muskler, som opleves ved KOL. Disse fremtræder som kardiovaskulære sygdomme, type-2 diabetes, osteoporose, lungecancer og muskelsvækkelse.\cite{dsam2016} Dertil har tobaksrygning samt dårlig livsstil betydning for udvikling af disse komorbiditeter \cite{McCarthy2015}. Foruden de nævnte komorbiditet, kan patienterne ligeledes opleve psykiske komorbiditet, såsom depression og angst, da patienterne ofte isolere sig på grund af generne ved KOL.\cite{dsam2016}

KOL kan ikke helbredes, og det er dertil ikke muligt at genvinde den tabte lungefunktion. Dog er det muligt at forhindre yderligere tab af lungefunktionen forårsaget af KOL samt lindre patienters symptomer.\cite{Basisbogen2016} Dette leder op til følgende initierende problemstilling.


\section{Initierende problemstilling}
\textit{Hvordan er nuværende diagnosticering og behandling af patienter med kronisk obstruktiv lungesygdom, og hvilke rehabiliteringsmuligheder kan tilbydes?}