\chapter{Indledning}
 
 **** SLET IKKE BEGYNDT ****

Kronisk obstruktiv lungesygdom (KOL)
KOL er på nuværende tidspunkt den tredje hyppigste dødsårsag på verdensplan[WHO]. Dertil er der i Danmark ca. 430.000 patienter med KOL med en årlig mortalitet på 3.500 patienter, hvilket gør KOL til den fjerde hyppigste dødsårsag i Danmark[basisbogen].


 **** SLET IKKE BEGYNDT ****

KOL udvikles over mange år, dog vil patienten ikke bemærke sygdommen førend lungefunktionen er markant nedsat. Dette betyder, at KOL og dens symptomer som regel først kommer til udtryk efter 50 års-alderen\cite{Lange2015}. Dette kan betyde, at patienter først opsøger en læge, når deres lungefunktion er halveret \cite{dsam2016}. 


\section{Initierende problemstilling}
\textit{Hvordan er diagnosticeringen og behandlingen af patienter med kronisk obstruktiv lungesygdom og hvilke rehabiliteringsmuligheder kan tilbydes?}