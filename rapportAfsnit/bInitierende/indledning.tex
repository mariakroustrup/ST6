\chapter{Indledning} 
Kronisk Obstruktiv lungesygdom (KOL) er den tredje største dødsårsag på verdensplan og er i Danmark den fjerde hyppigste. I Danmark lever ca. 430.000 mennesker med KOL og den årlige mortalitet er på 3.500. Dødeligheden er over 50 per 100.000 per år for både mænd og kvinder. Denne stiger til næsten 10 \% ved indlæggelse på grund af eksacerbationer i løbet af den første måned. 

KOL er en kronisk lungesygdom, der opstår ved inflammation i luftvejene og lungevævet, da bronkiernes vægge ødelægges og/eller luftvejene forsnævres. KOL udvikles over mange år, hvilket medvirker til, at nogle patienter ofte ikke vil bemærke sygdommen førend lungefunktionen er markant nedsat. Symptomerne på KOL er åndenød og hoste ved fysisk aktivitet. Den hyppigste årsag til KOL er tobaksrygning. 
Derudover er der en række komorbiditeter såsom kardiovaskulære sygdomme, type-2 diabetes, osteoporose, lungecancer, muskelsvækkelse samt angst og depression. Disse kan skyldes, at åndenød har medført svage perifere muskler på grund af et fysisk nedsat aktivitetsniveau.

Da KOL er en kronisk lungesygdom er det ikke muligt at genoprette den tabte lungefunktion, det er derfor vigtigt at opretholde den nuværende lungefunktion. Redskaber til at opretholde denne kan fås gennem rehabilitering. I rehabilitering fokuseres der på redskaber som tobaksafvænning, fysisk aktivitet samt vejledning om kost og medicinering, som alle kan bidrage til at mindske symptomer, eksacerbationer, hospitalsindlæggelser samt en forbedret livskvalitet. 
Ved fysisk aktivitet opnår patienter bedre udbytte af deres tilbageværende lungefunktion samt øger muskelfunktionen. Dette medvirker til at udsætte træthed samt en øget aktivitetstolerance. Foruden dette vil lungerne fremover belastes mindre ved fysisk aktivitet, hvilket kan forbedre patienters vejrtrækning. 

\section{Initierende problemstilling}
\textit{Hvordan er diagnosticeringen og behandlingen af patienter med kronisk obstruktiv lungesygdom og hvilke rehabiliteringsmuligheder kan tilbydes?}