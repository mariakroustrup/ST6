\chapter{Indledning}
Kronisk Obstruktiv lungesygdom (KOL) er en kronisk lungesygdom, hvor der opstår inflammation i luftvejene og lungevævet, da bronkiernes vægge ødelægges og/eller luftvejene forsnævres. KOL er den tredje største dødsårsag på verdensplan og er i Danmark den fjerde hyppigste. I Danmark lever ca. 430.000 mennesker med KOL og den årlige mortalitet er på 3.500. 

KOL udvikles over mange år, hvilket medvirker til at nogle patienter ofte ikke vil bemærke sygdommen førend lungefunktionen er markant nedsat. Dette betyder, at KOL og dens symptomer som regel først kommer til udtryk efter 50 års-alderen\cite{Lange2015}. Dette kan betyde, at patienter først opsøger en læge, når deres lungefunktion er halveret \cite{dsam2016}.

Da KOL er en kronisk lungesygdom er det ikke muligt at genoprette den tabte lungefunktion dog kan symptomerne, herunder XXXXXX  mindskes, hvilket kan medvirke til en forbedret livskvalitet.  




 der er forbundet med KOL. 





\section{Initierende problemstilling}
\textit{Hvordan er diagnosticeringen og behandlingen af patienter med kronisk obstruktiv lungesygdom og hvilke rehabiliteringsmuligheder kan tilbydes?}