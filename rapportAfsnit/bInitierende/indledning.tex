\chapter{Indledning} 
Kronisk Obstruktiv lungesygdom (KOL) er den tredje største dødsårsag på verdensplan og er i Danmark den fjerde hyppigste. I Danmark lever ca. 430.000 mennesker med KOL og den årlige mortalitet er på 3.500. KOL er en kronisk lungesygdom, der opstår ved inflammation i luftvejene og lungevævet, da bronkiernes vægge ødelægges og/eller luftvejene forsnævres. KOL udvikles over mange år, hvilket medvirker til at nogle patienter ofte ikke vil bemærke sygdommen førend lungefunktionen er markant nedsat. 

Da KOL er en kronisk lungesygdom er det ikke muligt at genoprette den tabte lungefunktion. Det er dog muligt at forhindre udviklingen af KOL og lindre symptomerne, såsom åndenød og hoste ved fysisk aktivtet, hvilket kan medvirke til en forbedret livskvalitet. Dette kan opnås ved tobaksafvænning, fysisk aktivitet, kostvejledning og medicin. 






\section{Initierende problemstilling}
\textit{Hvordan er diagnosticeringen og behandlingen af patienter med kronisk obstruktiv lungesygdom og hvilke rehabiliteringsmuligheder kan tilbydes?}