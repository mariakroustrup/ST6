\chapter{Metode}
Der er indsamlet litteratur for at opnå tilstrækkelig viden i forhold til at udvikle et hjælpemiddel til KOL patienter i rehabiliteringsfasen. Der er primært anvendt sekundær litteratur, herunder fagbøger eller analyse af problemstillinger, der er relevante i forhold til den initierende problemstilling. Derudover er der anvendt AAU-modellen til opbygning af rapporten, hvilket har medvirket til at få en struktureret opbygning af denne. 

\section{Videnindsamling}
Der er anvendt struktureret og ustruktureret søgning for at opnå tilstrækkelig viden. Den ustrukturede søgning er anvendt for at skabe en grundlæggende viden før projektskrivningen påbegyndes. Denne søgning foregik på Google og AUB, hvor mindre artikler samt mediciniske beregner har skabt en grundlæggende viden om KOL. Den strukturede søgning er anvendt til at besvare problemstillingen for projektet. Den strukturede søgning kan gøres ved brug af en model, hvor der er en fast struktur, som gentages for hver søgning. Til den strukturede søgning er anvendt AUB, PubMed med flere. 

\section{Opbygning af rapporten}
Rapporten er opbygget efter AAU-modellen som tager udgangspunkt i et initierende problemstilling, som er udarbejdet på baggrund af de spørgsmål der opstå gennem indledningen. Herefter belyses problemstillingen i problemanalysen, som indledes af metodeafsnit, herunder videnindsamling og rapportopbygning. Efter problemanalysen opsummeres de vigtigst pointer som leder frem til problemformuleringen. Projektet afgrænses i problemformulering til den primære patientgruppe. Der kan ske ændringer i denne på grund af ny videnindsamling. Opbygningen af rapporten fremgår af \ref{fig:AAUModel}.

\begin{figure} [H]
\centering
\includegraphics[width=0.3\textwidth]{figures/AAUModel}
\caption{Opbygning af rapport ud fra AAU-modellen.}
\label{fig:AAUModel}
\end{figure} 