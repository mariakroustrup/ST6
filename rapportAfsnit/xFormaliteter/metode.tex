Der er indsamlet litteratur for at opnå tilstrækkelig viden i forhold til at udvikle et hjælpemiddel til KOL-patienter med henblik på opretholdelse af resultaterne efter rehabiliteringsforløb. Der er primært anvendt sekundær litteratur, herunder fagbøger og videnskabelige artikler, der er relevante i forhold til den initierende problemstilling. For at opnå en struktureret opbygning af rapporten er AAU-modellen anvendt. 

\section{Vidensindsamling}
Der er anvendt struktureret og ustruktureret søgning for at opnå tilstrækkelig viden. Den ustrukturerede søgning er anvendt for at skabe en grundlæggende viden før påbegyndelse af projektskrivning. Denne søgning foregik på Google og AUB, hvor mindre artikler samt medicinske begreber har skabt en grundlæggende viden og forståelse om KOL. Den strukturerede søgning er anvendt til at besvare projektets problemstilling. I denne søgning er der anvendt AUB, PubMed med flere. Derudover er der udarbejdet en model for søgning for, at få en fast struktur over denne. Et eksempel på dette fremgår af \autoref{tab:viden}.

\begin{table}[H]
\centering
\label{tab:viden}
\begin{tabular}{|l|l|}
\hline
Emne & Søgeord                                 \\ \hline
KOL & Kronisk obstruktiv lungesygdom, KOL, Chronic Obstructive Pulmonary Disease, COPD...\fxnote{Skriv mere til her} \\ \hline
\end{tabular}
\end{table}

\section{Opbygning af rapporten}
Denne rapport er opbygget efter AAU-modellen, der tager udgangspunkt i en problembaseret tilgang. AAU-modellen fremgår af \autoref{fig:AAUModel}. Rapporten indledes med en bred litteratursøgning, hvor det initierende problem opstilles. Dette problem undersøges i problemanalysen, hvor en afgrænsning til problemformuleringen forekommer. Problemformuleringen forsøges envidere besvaret i problemløsningen, der udformes efter vandfaldsmodellen, jf. \autoref{sec:vandfald}. Herunder vil problemløsningen analyseres, hvorved der ligeledes opstilles kravspecifikationer.  Promløsningen vil derudover designes, implementeres og testes. Efterfølgende vil en diskussion af problemanalysen og problemløsningen lede op til besvarelse af problemformuleringen, der forekommer i en samlet konklussion for projektet. Til sidst afsluttes projektet med en perspektivering.


\begin{figure} [H]
\centering
\includegraphics[width=0.3\textwidth]{figures/AAUModel}
\caption{Opbygning af rapport ud fra AAU-modellen. \fxnote{dette er blot et udkast, hvis vi vælger at have implementering og test under et, skal den laves om.}}
\label{fig:AAUModel}
\end{figure} 
