\chapter{Metode}
Der er indsamlet litteratur for at opnå tilstrækkelig viden i forhold til at udvikle et hjælpemiddel til KOL patienter i rehabiliteringsfasen. Der er primært anvendt sekundær litteratur, herunder fagbøger eller analyse af problemstillinger, der er relevante i forhold til den initierende problemstilling. Derudover er AAU-modellen anvendt med henblik på at opnå en struktureret opbygning. 

\section{Vidensindsamling}
Der er anvendt struktureret og ustruktureret søgning for at opnå tilstrækkelig viden. Den ustrukturerede søgning er anvendt for at skabe en grundlæggende viden før projektskrivningen påbegyndes. Denne søgning foregik på Google og AUB, hvor mindre artikler samt medicinske begreber har skabt en grundlæggende viden og forståelse om KOL. Den strukturerede søgning er anvendt til at besvare problemstillingen for projektet. Til denne søgning er der anvendt AUB, PubMed med flere. For at få fast struktur over søgning, således denne kan gentages for hver søgning er der udarbejdet en model for søgningen. Et eksempel på dette fremgår af \ref{tab:viden}. \fxnote{denne tabel er blot en idé i forhold til at dokumentere vores litteratursøgning}

\begin{table}[H]
\centering
\label{tab:viden}
\begin{tabular}{|l|l|}
\hline
Ord & Ordliste                                 \\ \hline
KOL & KOL, Kronisk obstruktiv lungesygdom osv. \\ \hline
\end{tabular}
\end{table}

\section{Opbygning af rapporten}
Rapporten er opbygget efter AAU-modellen, som fremgår af \ref{fig:AAUModel}, som tager udgangspunkt i et initierende problemstilling, som er udarbejdet på baggrund af de spørgsmål der opstod gennem indledningen. Herefter belyses problemstillingen i problemanalysen, som indledes af metodeafsnit, herunder vidensindsamling og rapportopbygning. Efter problemanalysen opsummeres de vigtigst pointer som leder frem til problemformuleringen. Projektet afgrænses i problemformulering til den primære patientgruppe. Der kan ske ændringer i denne på grund af ny vidensindsamling. .

\begin{figure} [H]
\centering
\includegraphics[width=0.3\textwidth]{figures/AAUModel}
\caption{Opbygning af rapport ud fra AAU-modellen. \fxnote{dette er blot et udkast, hvis vi vælger at have implementering og test under et, skal den laves om.}}
\label{fig:AAUModel}
\end{figure} 

Efter problemafgrænsningen belyses metoder til at besvarelse af  problemformuleringen, hvor problemløsningen efterfølgende vil analyseres, design, implementeres og testes. Til sidst diskuteres, konkluderes og perspektiveres problemløsningen og problemformuleringen i syntesen. 

