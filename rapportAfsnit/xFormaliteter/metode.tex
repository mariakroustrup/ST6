I dette kapitel beskrives metoden anvendt til opbygning af rapporten med henblik på at opnå struktur. Herudover beskrives, hvordan litteratur er indsamlet for at opnå tilstrækkelig viden om KOL. Metode til problemløsning forekommer senere af et andet metodeafsnit.

\section{Opbygning af rapporten}
Denne rapport er opbygget efter AAU-modellen, der tager udgangspunkt i en problembaseret tilgang. AAU-modellen fremgår af \autoref{fig:AAUModel}. Rapporten indledes med en bred litteratursøgning, hvor det initierende problem opstilles. Dette problem undersøges i problemanalysen, hvor en afgrænsning til problemformuleringen forekommer. Problemformuleringen forsøges endvidere besvaret i problemløsningen, der udformes efter Unified Proces, jf. \autoref{sec:UP}. Herunder vil løsningen analyseres, hvortil der opstilles kravspecifikationer. Et løsningsforslag vil herefter designes, implementeres og testes. Efterfølgende vil en diskussion af problemanalysen og problemløsningen lede op til besvarelse af problemformuleringen, der forekommer i en samlet konklusion for projektet. Til sidst afsluttes projektet med en perspektivering.

\begin{figure} [H]
\centering
\includegraphics[width=0.5\textwidth]{figures/AAUModel}
\caption{Opbygning af rapport ud fra AAU-modellen.}
\label{fig:AAUModel}
\end{figure} 

\section{Vidensindsamling}
Der er anvendt ustruktureret og struktureret søgning for at opnå tilstrækkelig viden. Den ustrukturerede søgning er anvendt for at skabe en grundlæggende viden før påbegyndelse af projektskrivning. Denne søgning foregik på Google og Primo, hvor artikler samt medicinske begreber har skabt en grundlæggende viden og forståelse om KOL. Den strukturerede søgning er anvendt til at besvare projektets problemstilling. I denne søgning er der anvendt Primo, PubMed med flere. Derudover er der udarbejdet en model for søgning for at få en fast struktur over denne. Et eksempel på dette fremgår af \autoref{tab:sogeord}.

\begin{table}[H]
\centering
\begin{tabular}{|l|l|}
\hline
\textbf{Emne}                  & \textbf{Søgeord}                                                                                                                                   \\ \hline
Kronisk obstruktiv lungesygdom & \begin{tabular}[c]{@{}l@{}}KOL, Chronic Obstructive Pulmoray Disease,\\ COPD, Diagnose, Behandling, Treatment,\\ Incidens, Prævalens. \end{tabular} \\ \hline
\end{tabular}
\caption{Eksempel på anvendte søgeord for KOL. Disse søgeord er anvendt alene samt i kombination.}
\label{tab:sogeord}
\end{table}

