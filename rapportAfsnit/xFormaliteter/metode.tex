Der er indsamlet litteratur for at opnå tilstrækkelig viden i forhold til at udvikle et hjælpemiddel til KOL-patienter efter rehabiliteringsforløbet. Der er primært anvendt sekundær litteratur, herunder fagbøger eller analyse af problemstillinger, der er relevante i forhold til den initierende problemstilling. For at opnå en struktureret opbygning af rapporten er AAU-modellen anvendt. 

\section{Vidensindsamling}
Der er anvendt struktureret og ustruktureret søgning for at opnå tilstrækkelig viden. Den ustrukturerede søgning er anvendt for at skabe en grundlæggende viden før påbegyndelse af projektskrivning. Denne søgning foregik på Google og AUB, hvor mindre artikler samt medicinske begreber har skabt en grundlæggende viden og forståelse om KOL. Den strukturerede søgning er anvendt til at besvare projektets problemstilling. I denne søgning er der anvendt AUB, PubMed med flere. Derudover er der udarbejdet en model for søgning for, at få en fast struktur over denne. Et eksempel på dette fremgår af \ref{tab:viden}. \fxnote{denne tabel er blot en idé i forhold til at dokumentere vores litteratursøgning}

\begin{table}[H]
\centering
\label{tab:viden}
\begin{tabular}{|l|l|}
\hline
Ord & Ordliste                                 \\ \hline
KOL & Kronisk obstruktiv lungesygdom, KOL, Chronic Obstructive Pulmonary Disease, COPD... \\ \hline
\end{tabular}
\end{table}

\section{Opbygning af rapporten}
Rapporten er opbygget efter AAU-modellen, som fremgår af \ref{fig:AAUModel}. Denne tager udgangspunkt i et initierende problemstilling, som er udarbejdet på baggrund af de spørgsmål der opstod gennem indledningen. Herefter belyses problemstillingen i problemanalysen, som indledes af et metodeafsnit, herunder vidensindsamling og rapportopbygning. Efter problemanalysen opsummeres de vigtigst pointer som leder frem til problemformuleringen. Projektet afgrænses i problemformuleringen til den primære målgruppe samt problemet, som ønskes at belyses gennem problemløsningen. 

Efter problemafgrænsningen belyses de metoder der anvendes for at besvarelse problemformuleringen. Efterfølgende vil løsningen til problemet analyseres, designes, implementeres og testes. Til sidst diskuteres, konkluderes og perspektiveres problemløsningen og problemformuleringen i syntesen. 

\begin{figure} [H]
\centering
\includegraphics[width=0.3\textwidth]{figures/AAUModel}
\caption{Opbygning af rapport ud fra AAU-modellen. \fxnote{dette er blot et udkast, hvis vi vælger at have implementering og test under et, skal den laves om.}}
\label{fig:AAUModel}
\end{figure} 
