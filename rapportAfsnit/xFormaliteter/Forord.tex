% !TeX spellcheck = da_DK
\chapter*{Forord og læsevejledning}

\section*{Forord}
Dette bachelorprojekt er udarbejdet af gruppe 17gr6403 på ingeniøruddannelsen Sundhedsteknologi på Aalborg Universitet i perioden 1. februar til 30. maj 2017. Projektet tager udgangspunkt i det overordnede tema \textit{Design af sundhedsteknologiske systemer} og projektforslaget \textit{Udvikling af KOL patientens nye bedste ven - den smarte KOL trænings-app!}, som er stillet af Lars Pilegaard Thomsen. 
Læringsmålet for dette projekt er ifølge studieordningen: \textit{Bachelorprojektet er afslutningen på bacheloruddannelsen og den studerende skal kunne demonstrere evner, som er relevante for arbejdsmarkedet og for en videre videnskabelig uddannelse} \cite{Studieordning2014}.

Projektgruppen retter tak til hovedevejleder Lars Pilegaard Thomsen for vejledning og feedback gennem projektperioden.

\section*{Læsevejledning}
Projektet er delt op i tre dele, herunder problemanalyse, problemløsning og synopsis. I problemanalysen analyseres den opstillede problemstilling, hvor problemløsningen omhandler analyse, design, implementering og test af et system. Der er udarbejdet to metodeafnsit, hvoraf det første beskriver strukturen af rapporten samt vidensindsamling. Det andet metodeafsnit omfatter metoden anvendt i problemløsningen. Projektet afsluttes med en syntese, der omfatter diskussion, konklusion samt perspektivering. Dette efterfølges af litteraturliste samt bilag. 

I dette projekt anvendes Vancouver-metoden til håndtering af kilder. De anvendte kilder nummereres fortløbende i kantede parenteser. Er kilderne angivet før punktum i en sætning henvender denne sig til den pågældende sætning. Er kilden angivet efter punktum henvender denne sig til det foregående afsnit. I litteraturlisten ses kilderne, der er angivet med forfatter, titel og årstal. Forkortelser i rapporten er første gang skrevet ud, efterfulgt af forkortelsen angivet i parentes. Herefter anvendes forkortelsen fremadrettet i rapporten. Hvis centrale elementer fra figurer yderligere er beskrevet markeres dette med kursiv. 

Rapporten er udarbejdet i \LaTeX, og app'en er udviklet i Android Studio version 2.3.1.
Af nedenstående link forekommer en demonstrationsvideo af den udarbejdet app. \fxnote{HUSK LINK!}