I Danmark er kronisk obstruktiv lungesygdom (KOL) den tredje hyppigste dødsårsag, hvor op mod 430.000 lider af dette. KOL er en kronisk sygdom, der gradvist nedsætter lungefunktionen. Da KOL ikke kan behandles, tilbydes disse patienter rehabilitering, hvor der rådgives inden for sygdommen, tobaksophør, kost, medicinering og motion. Patienter opnår gennem rehabiliteringsforløb en symptomlindrende effekt, bedre udnyttelse af den tilbageværende lungefunktion samt et bedre fysisk funktionsniveau. Studier viser dog, at nogle patienter ikke kan opretholde de gode resultater efter endt rehabiliteringsforløb, idet de falder tilbage til vante vaner og rutiner. På baggrund af denne problemstilling er der udarbejdet en app, der har til formål at vejlede samt motivere KOL-patienter til regelmæssig træning. App’en vejleder i forhold til en anbefalet træningstid, der er tilpasset den individuelle KOL-patient ud fra patientens tilstand. Der er implementerede funktionaliteter, der har til formål at virke motiverende for patienten. Disse indebærer blandt andet muligheden for at opnå belønninger, fælles interaktioner samt notifikationer. 
 
App’en er blot en prototype, hvorfor der skal foretages ændringer, hvis denne skal implementeres i praksis. 

