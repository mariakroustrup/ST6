Chronic obstructive pulmonary disease (COPD) is the fourth most common cause of death in Denmark, where up to 430,000 people suffers from it. 
COPD is a disease that causes the patient to gradually lose lung function over time, to where there is no possibility of restoring it again.
Since there is no treatment, COPD patients is offered to join a pulmonary rehabilitation program, where they get the chance to learn about COPD, smoking discontinuation, diet, medication and exercise. 
Pulmonary rehabilitation allows the patients to achieve a relief in symptoms, through regular exercise, in which they get a better utilization of the remaining lung function and a better physical function level. 
However, studies show that some of these patients are not able to maintain the symptom relieving effects as they fall back to previous habits and routines after 6 to 12 months post-rehabilitation. 
In this project an app has been developed as a means of addressing this issue. The intention of the app is to guide and motivate COPD patients to maintain regular exercise during post-rehabilitation. 
Based on the individual COPD patient's condition, the app adapts the timeline of the recommended exercise, so it will be best suitable for the patient. 
The app possess different functionalities, as a way of motivating the patients and remind them of regular exercise, like daily notifications and the ability to obtain achievements based on different aspects of exercise and joint interaction in which users of the app can follow each other's progress.
However the app is currently a prototype, in which changes must be made before it can be implemented and tested in practice.