I dette kapitel opstilles krav til en app for at kunne udvikle et system, der opfylder problemformuleringen. Først beskrives systemet ud fra, hvad der ønskes, at det skal kunne, hvorudfra funktionelle og non-funktionelle krav opstilles.

\section{Systembeskrivelse} \label{sec:systembeskrivelse}
I dette projekt udvikles en app, der har til formål at kunne anvendes af KOL-patienter som et redskab til at opretholde den kondition, som er opnået under rehabiliteringsforløb. Derudover skal app’en motivere KOL-patienter til regelmæssig motion og eventuelt forbedre deres kondition yderligere. Konditionen opretholdes ved forskellige træningsformer, herunder konditionstræning, styrketræning og vejrtrækningsøvelser, hvilket har en symptom reducerende effekt jf. \autoref{sec:rehabilitering}. 

KOL-patienter introduceres til app’en på rehabiliteringshold for, at de kan opnå kendskab til app’ens formål samt funktioner. Registreringen foregår i samarbejde med sundhedspersonalet for således at sikre, at det kun er KOL-patienter, der er tilmeldt rehabiliteringshold, som har adgang til app’en. Ved registrering oprettes KOL-patienter med medlemsID, navn, efternavn, kodeord samt kategorisering i forhold til deres sygdom. 

Der er forskel på, hvor meget fysisk aktivitet KOL-patienter kan udføre, og der er derved forskel i varighed og distance af træningen. For at tage højde for dette udvælges træningen på baggrund af den enkelte patients angivet parametre. Disse parametre skal reducere risikoen for negative oplevelser ved anvendelse af app’en og tage højde for daglige variationer. Dette forsøges undgået ved kategorisering af KOL-patienter efter ABCD jf. \autoref{PROBLEMANALYSE}, daglige helbredstilstande samt evalueringer af træning og biologiske målinger.  

For at hjælpe KOL-patienter med vedligeholdelse af den daglige træning skal app’en virke motiverende for patienterne. Dette gøres blandt andet ved at gøre det muligt for KOL-patienter at følge sin egen udvikling via app’en. App’en skal desuden gøre KOL-patienter opmærksom på, hvis de ikke har været aktive på app'en i længere tid. For at øge motivation hos KOL-patienter skal de derudover kunne opnå virtuelle belønninger ved at udføre træningssessioner.

Som nævnt i afsnit \ref{sec:efterRehabilitering}, er det sociale fællesskab en væsentlig faktor i opretholdelse af resultaterne fra rehabiliteringsforløb. Ved at indføre denne faktor i app’en kan dette være med til at motivere KOL-patienter til vedligeholdelse af den forbedrede livsstil. Dette gør det desuden muligt at gøre patienter opmærksom på, at andre brugere af app’en har udført en træningssession, hvilket kan virke motiverende.
Sundhedsfagligt personale skal kunne tilgå KOL-patienters resultater, så de kan følge med i udviklingen. De har herved mulighed for at informere patienter om, hvorvidt de træner for lidt eller om de gør et godt arbejde, hvilket også kan have en motiverende effekt på KOL-patienter.

\subsection{Funktionelle krav} \label{sec:funktionellekrav}
Ud fra informationer fundet gennem problemanalysen i \autoref{cha:problemanalyse} og beskrivelsen af systemet i \autoref{sec:systembeskrivelse} er nedenstående funktionelle krav opstillet. 



\noindent
%\textbf{Database:}  
\begin{itemize}
\item Systemet skal kunne oprette forbindelse til databasen
	\\
	\textit{Dette er nødvendigt for at gemme, hente og redigere data}

\item Systemet skal kunne gemme og hente data i databasen
	\\
	\textit{Dette er nødvendigt for at tilgå tidligere resultater for både patienter og sundhedspersonale}	
\item Sundhedspersonalet skal kunne oprette brugere i databasen
	\\
	\textit{Dette er nødvendigt for, at flere KOL-patienter kan anvende app'en}
\end{itemize}
	
\noindent	
%\textbf{Systemet:}
\begin{itemize}
\item KOL-patienter skal kunne redigere brugeroplysninger
	\\
	\textit{Dette er nødvendigt for, at patienter kan ændre adgangskode og kategorisering}
\item KOL-patienter skal kunne angive deres daglige helbredstilstand 
\\
\textit{Dette er nødvendigt for at tilpasse træningen for den enkelte patient}
\item KOL-patienter skal kunne evaluere hver træning
\\
	\textit{Dette er nødvendigt for at tilpasse træningen for den enkelte patient }
\item KOL-patienter skal kunne tilgå deres egne resultater
	\\
	\textit{Dette er nødvendigt for, at patienter kan se deres træningens udvikling}
\item KOL-patienten skal kunne log ind og log ud
	\\
	\textit{Dette er nødvendigt for at tilgå og sikre patienters individuelle data}
\item Systemet skal kunne kategorisere KOL-patienter i ABCD
\\
	\textit{Dette er nødvendigt for at kunne tilpasse træningen efter den enkelte KOL-patient}
\item Systemet skal kunne sende notifikationer og give virtuelle belønninger 
	\\
	\textit{Dette er nødvendigt for at kunne motivere patienter til at udføre træning}
\item Systemet skal muliggøre interaktion mellem KOL-patienter
	\\
	\textit{Dette er nødvendigt for motivering mellem KOL-patienter samt skabe fællesskab}
	\\
	\textit{Dette er nødvendigt for at patienter kan tilgå hinandens virtuelle belønninger}
	%\\
	%\textit{Dette er nødvendigt for at kunne give notifikationer, hvis andre patienter træner}	
\item Systemet skal tilgås med et medlemsID eller brugernavn og kodeord
	\\
\textit{Dette er nødvendigt for at sikre, at KOL-patienter har deltaget i et rehabiliteringsforløb samt adskille patienters data}
\item Systemet skal kunne synkronisere med biologiske måleenheder. 
\\
\textit{Dette er nødvendigt for at måle biologiske målinger under træning}
\end{itemize}


\subsection{Non-funktionelle krav}
De non-funktionelle krav er opstillet ud fra den overbevisning, at dette ikke er krav til systemets funktionalitet, men stadig er relevant i relation til den bedste udbredelse, brugervenlighed og brugeroplevelse. 

\begin{itemize}
\item Systemet skal kunne fungere på en smartphone eller tablet med android og bluetooth. 
\item Systemet skal være brugervenligt
	\\
	\textit{Dette er nødvendigt for at KOL-patienter let kan orientere sig}
\end{itemize}



%I dette kapitel opstilles krav til en app for at kunne udvikle et system, der opfylder problemformuleringen. Først beskrives systemet ud fra, hvad der ønskes, at det skal kunne, hvorudfra funktionelle og non-funktionelle krav opstilles.
%
%\section{Systembeskrivelse} \label{sec:systembeskrivelse}
%I dette projekt udvikles en app, der har til formål at kunne anvendes af KOL-patienter som et redskab til opretholdelse af de gavnlige effekter, der opnås under rehabiliteringsforløb.
%
%Under rehabiliteringsforløb lærer KOL-patienter forskellige træningsøvelser, som har en symptomlindrende effekt, jf. afsnit \ref{sec:rehabilitering}. App’en skal foreslå træningsøvelser, som KOL-patienterne kan udføre, så de fortsat får udført fysisk aktivitet i hjemmet efter endt rehabiliteringsforløb. Disse træningsøvelser skal udvælges på baggrund af de øvelser, der udføres under rehabiliteringsforløb, således at KOL-patienter allerede har erfaringer og kendskab til de øvelser, som app’en foreslår.
%Der er forskel på, hvor meget fysisk aktivitet forskellige KOL-patienter kan udføre, og der er derved forskel i varighed og intensitet af de træningsøvelser. For at tage højde for dette under valg af træningsøvelser, udvælges øvelserne på baggrund af den enkelte patients sværhedsgrad af KOL. Sværhedsgraden bestemmes ud fra ABCD-kategoriseringen, jf. \ref{sec:klassifikation}, som patienter skal angive, når de registreres som brugere første gang app'en anvendes. 
%For at reducere risikoen for, at KOL-patienter får negative oplevelser ved anvendelse af app’en, hvilket eksempelvis ville kunne ske, hvis der foreslås træningsøvelser på for højt niveau i forhold til patientens tilstand, skal app’en kunne tage højde for de daglige variationer i patientens sygdom. Dette imødekommes ved, at patienter angiver sit helbred til app'en den pågældende dag.
%Ud fra dette tilpasses træningsøvelserne, som KOL-patienterne bliver foreslået. Efter en udført
%træningssession skal patienten evaluere træningen i forhold til sværhedsgraden af træningen,
%så denne evaluering kan medtages i udvælgelsen af den efterfølgende dags anbefalede træning. Herved kan træningen tilpasses den enkelte KOL-patients tilstand.
%For at hjælpe KOL-patienter med vedligeholdelse af den daglige træning skal app’en virke motiverende for patienterne. Dette gøres blandt andet ved at gøre det muligt for KOL-patienter at følge sin egen udvikling via app’en. App’en skal desuden gøre KOL-patienter opmærksom på, hvis de ikke har været aktive på app'en i længere tid. For at øge motivation hos KOL-patienter skal de derudover kunne opnå virtuelle belønninger ved at udføre træningssessioner.
%
%Som nævnt i afsnit \ref{sec:efterRehabilitering}, er det sociale fællesskab en væsentlig faktor i opretholdelse af resultaterne fra rehabiliteringsforløb. Ved at indføre denne faktor i app’en kan dette være med til at motivere KOL-patienter til vedligeholdelse af den forbedrede livsstil. Dette gør det desuden muligt at gøre patienter opmærksom på, at andre brugere af app’en har udført en træningssession, hvilket kan virke motiverende.
%Sundhedsfagligt personale skal kunne tilgå KOL-patienters resultater, så de kan følge med i udviklingen. De har herved mulighed for at informere patienter om, hvorvidt de træner for lidt eller om de gør et godt arbejde, hvilket også kan have en motiverende effekt på KOL-patienter.
%
%\subsection{Funktionelle krav}
%Ud fra informationer fundet gennem problemanalysen i \autoref{cha:problemanalyse} og beskrivelsen af systemet i \autoref{sec:systembeskrivelse} er nedenstående funktionelle krav opstillet. 
%  
%\begin{itemize}
%\item Systemet skal kunne oprette forbindelse til databasen
%	\\
%	\textit{Dette er nødvendigt for at gemme, hente og redigere data}
%
%\item Systemet skal kunne gemme data i databasen
%	\\
%	\textit{Dette er nødvendigt for at tilgå tidligere resultater}
%
%\item Systemet skal kunne hente data fra databasen
%	\\
%	\textit{Dette er nødvendigt for at tilgå tidligere resultater for både patienter og sundhedspersonale}	
%	
%\item Systemet skal kunne oprette nye brugere 
%	\\
%	\textit{Dette er nødvendigt for, at flere KOL-patienter kan anvende app'en}
%
%\item Systemet skal kunne redigere brugeroplysninger
%	\\
%	\textit{Dette er nødvendigt, hvis patienters tilstand ændres}
%
%\item Systemet skal have en login og logout funktion
%	\\
%	\textit{Dette er nødvendigt for at sikre patienters individuelle data}
%
%\item Systemet skal kunne kategorisere og vurdere  KOL-patienters helbredstilstand 
%	\\
%	\textit{Dette er nødvendigt for at kunne tilpasse træningssæt efter den enkelte KOL-patient}
%
%\item Systemet skal kunne sende notifikationer og give virtuelle belønninger 
%	\\
%	\textit{Dette er nødvendigt for at kunne motivere patienter til at udføre træning}
%
%\item Systemet skal kunne interagere med andre KOL-patienter
%	\\
%	\textit{Dette er nødvendigt for motivering mellem KOL-patienter samt skabe fællesskab}
%	\\
%	\textit{Dette er nødvendigt for at patienter kan tilgå hinandens virtuelle belønninger}
%	\\
%	\textit{Dette er nødvendigt for at kunne give notifikationer, hvis andre patienter træner}
%
%\end{itemize}
%
%\subsection{Non-funktionelle krav}
%De non-funktionelle krav er opstillet ud fra en den overbevisning, at dette ikke er krav til systemets funktionalitet, men stadig er relevant i relation til den bedste udbredelse, brugervenlighed og brugeroplevelse. 
%
%\begin{itemize}
%\item Systemet skal kunne fungere på en smartphone eller tablet 
%\item Systemet skal være brugervenligt
%	\\
%	\textit{Dette er nødvendigt, da KOL-patienter ofte er ældre}
%	
%\item Systemet skal tilgås med et medlemsnummer, navn og kodeord
%	\\
%	\textit{Dette er nødvendigt for at sikre, at KOL-patienter har deltaget i et rehabiliteringsforløb}
%	\\
%	\textit{Dette er nødvendigt for at adskille patienters data}
%\end{itemize}