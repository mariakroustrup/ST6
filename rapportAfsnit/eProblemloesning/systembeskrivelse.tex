I dette kapitel beskrives funktionaliteten af den ønskede app. På baggrund af dette opstilles funktionelle samt non-funktionelle krav. Herefter er systemet beskrevet ved hjælp af et use case diagram, hvortil de enkelte funktionaliteter er beskrevet yderligere. 

\section{Systembeskrivelse} \label{sec:systembeskrivelse}
I dette projekt udvikles en app, der har til formål at hjælpe KOL-patienter til at opretholde regelmæssig motion efter et endt rehabiliteringsforløb. App'en skal kunne håndtere forskellige træningsformer herunder konditions- samt styrketræning og vejrtrækningsøvelser, hvilket alle har symptomreducerende effekt, jf. \autoref{sec:rehabilitering}. Ligeledes skal app'en kunne håndtere forskellige træningstyper, som for eksempel gå, løbe eller cykle.

KOL-patienter introduceres samt registreres i app'en i forbindelse med deres rehabiliteringsforløb. 
Dette skal sikre, at det kun er KOL-patienter, der er tilmeldt rehabiliteringshold, som kan anvende app'en til træning. Ved registrering oprettes KOL-patienter med medlemsID, fornavn, efternavn samt adgangskode.  

Der er forskel på, hvor meget fysisk aktivitet KOL-patienter kan udføre, og der skal derved være forskel i varighed af den træning som app'en foreslår. Dertil skal app'en kunne tilpasse træningsniveau ud fra den enkelte patients parametre. Disse parametre består af kategoriseringen af KOL-patienter efter ABCD, jf. \autoref{cha:problemanalyse}, daglige helbredstilstande, der skal tage højde for dag til dag variationer samt tidligere evalueringer fra lignende træningstype. Dette medvirker til, at app'en er henvendt specifik til KOL-patienter i modsætning til andre træningsapp's, der henvender sig til hele befolkningen. 

Under selve træningen monitoreres træningen ved brug af timer og GPS. Monitoreringen er med til at vejlede patienten til at følge det valgte træningsniveau. 

For at hjælpe KOL-patienter med vedligeholdelse af den daglige træning skal app’en virke motiverende for patienterne. Dette gøres blandt andet ved, at KOL-patienter kan opnå virtuelle belønninger ved at udføre gentagne eller forskellige træningsformer. Desuden skal app'en informere, hvis KOL-patienter ikke har udført træning med app'en i længere tid. \citep{Gade2007, Tricomi2016}

Som nævnt i afsnit \ref{sec:efterRehabilitering}, er det sociale fællesskab en væsentlig faktor for at opretholde resultaterne fra rehabiliteringsforløb. Dette gøres ved, at KOL-patienter kan følge og tilgå andre KOL-patienters virtuelle belønninger, hvormed dette motiverer patienterne til vedligeholdelse af den forbedrede livsstil. Derudover skal sundhedsfagligt personale kunne tilgå KOL-patienters resultater i en database, så de kan følge med i patienters udvikling. De har herved mulighed for at informere patienter om deres indsats, hvilket ligeledes kan have en motiverende effekt.\citep{Gade2007, Tricomi2016}


\section{Kravspecifikationer} \label{sec:funktionellekrav}
På baggrund af systembeskrivelsen er funktionelle og non-funktionelle krav til app'en opstillet. De funktionelle krav beskriver, hvilke funktionaliteter app'en skal have. De non-funktionelle krav er opstillet ud fra overbevisningen om, at det ikke er krav til systemets funktionalitet, men er relevant i relation til brugervenlighed og brugeroplevelse. 


\subsubsection{Funktionelle krav}

\noindent 
\begin{itemize}
\item Brugere skal kunne oprettes i en database
	\\
	\textit{Dette er nødvendigt for, at brugere kan anvende app'en}
	\item Systemet skal kunne gemme og hente data i en database
\\
\textit{Dette er nødvendigt for, at brugere kan tilgå brugerdata}
\item Brugere skal kunne logge ind med et personligt medlemsID og adgangskode
	\\
	\textit{Dette er nødvendigt for at tilgå og sikre, at brugere har deltaget i et rehabiliteringsforløb samt adskille brugernes data}
\item Systemet skal kunne kategorisere brugere i ABCD på baggrund af CATscore og  antallet af indlæggelser på grund af KOL. 
	\\
	\textit{Dette er nødvendigt for at kunne tilpasse træningen efter den enkelte bruger}
\item Brugere skal kunne angive deres daglige helbredstilstand 
	\\
\textit{Dette er nødvendigt for tage højde for daglige variationer og derved tilpasse træningen for den enkelte bruger}
\item Systemet skal kunne måle tid og afstand
	\\
\textit{Dette er nødvendigt for at monitorere træningen}
\item Brugere skal kunne evaluere hver træning
	\\
\textit{Dette er nødvendigt for at tilpasse træningen efter den enkelte bruger}		
\item Systemet skal kunne sende en notifikation, hvis brugere ikke har trænet før klokken 15. 
	\\
	\textit{Dette er nødvendigt for at kunne motivere brugere til at udføre træning}
	
	\item Systemet skal kunne give virtuelle belønninger 
	\\
	\textit{Dette er nødvendigt for at kunne motivere brugere til at udføre træning}
\item Brugere skal kunne følge andre brugere 
	\\
	\textit{Dette er nødvendigt for at skabe fællesskab samt gøre det muligt for brugere at tilgå hinandens virtuelle belønninger, hvilket skal øge brugeres motivation}
\item Brugere skal kunne redigere deres adgangskode
	\\
	\textit{Dette er nødvendigt for, at brugere skal kunne gøre deres adgangskode personlig}
\item Brugere skal kunne log ud af app'en
	\\
	\textit{Dette er nødvendigt for sikre brugerens individuelle data}
\end{itemize}


\subsubsection{Non-funktionelle krav}

\begin{itemize}
\item Systemet skal visualiseres på en smartphone eller tablet med android 
\item Systemet skal være brugervenligt
	\\
	\textit{Dette er nødvendigt for at sikre let orientering i app'en}
\end{itemize}



%I dette kapitel opstilles krav til en app for at kunne udvikle et system, der opfylder problemformuleringen. Først beskrives systemet ud fra, hvad der ønskes, at det skal kunne, hvorudfra funktionelle og non-funktionelle krav opstilles.
%
%\section{Systembeskrivelse} \label{sec:systembeskrivelse}
%I dette projekt udvikles en app, der har til formål at kunne anvendes af KOL-patienter som et redskab til opretholdelse af de gavnlige effekter, der opnås under rehabiliteringsforløb.
%
%Under rehabiliteringsforløb lærer KOL-patienter forskellige træningsøvelser, som har en symptomlindrende effekt, jf. afsnit \ref{sec:rehabilitering}. App’en skal foreslå træningsøvelser, som KOL-patienterne kan udføre, så de fortsat får udført fysisk aktivitet i hjemmet efter endt rehabiliteringsforløb. Disse træningsøvelser skal udvælges på baggrund af de øvelser, der udføres under rehabiliteringsforløb, således at KOL-patienter allerede har erfaringer og kendskab til de øvelser, som app’en foreslår.
%Der er forskel på, hvor meget fysisk aktivitet forskellige KOL-patienter kan udføre, og der er derved forskel i varighed og intensitet af de træningsøvelser. For at tage højde for dette under valg af træningsøvelser, udvælges øvelserne på baggrund af den enkelte patients sværhedsgrad af KOL. Sværhedsgraden bestemmes ud fra ABCD-kategoriseringen, jf. \ref{sec:klassifikation}, som patienter skal angive, når de registreres som brugere første gang app'en anvendes. 
%For at reducere risikoen for, at KOL-patienter får negative oplevelser ved anvendelse af app’en, hvilket eksempelvis ville kunne ske, hvis der foreslås træningsøvelser på for højt niveau i forhold til patientens tilstand, skal app’en kunne tage højde for de daglige variationer i patientens sygdom. Dette imødekommes ved, at patienter angiver sit helbred til app'en den pågældende dag.
%Ud fra dette tilpasses træningsøvelserne, som KOL-patienterne bliver foreslået. Efter en udført
%træningssession skal patienten evaluere træningen i forhold til sværhedsgraden af træningen,
%så denne evaluering kan medtages i udvælgelsen af den efterfølgende dags anbefalede træning. Herved kan træningen tilpasses den enkelte KOL-patients tilstand.
%For at hjælpe KOL-patienter med vedligeholdelse af den daglige træning skal app’en virke motiverende for patienterne. Dette gøres blandt andet ved at gøre det muligt for KOL-patienter at følge sin egen udvikling via app’en. App’en skal desuden gøre KOL-patienter opmærksom på, hvis de ikke har været aktive på app'en i længere tid. For at øge motivation hos KOL-patienter skal de derudover kunne opnå virtuelle belønninger ved at udføre træningssessioner.
%
%Som nævnt i afsnit \ref{sec:efterRehabilitering}, er det sociale fællesskab en væsentlig faktor i opretholdelse af resultaterne fra rehabiliteringsforløb. Ved at indføre denne faktor i app’en kan dette være med til at motivere KOL-patienter til vedligeholdelse af den forbedrede livsstil. Dette gør det desuden muligt at gøre patienter opmærksom på, at andre brugere af app’en har udført en træningssession, hvilket kan virke motiverende.
%Sundhedsfagligt personale skal kunne tilgå KOL-patienters resultater, så de kan følge med i udviklingen. De har herved mulighed for at informere patienter om, hvorvidt de træner for lidt eller om de gør et godt arbejde, hvilket også kan have en motiverende effekt på KOL-patienter.
%
%\subsection{Funktionelle krav}
%Ud fra informationer fundet gennem problemanalysen i \autoref{cha:problemanalyse} og beskrivelsen af systemet i \autoref{sec:systembeskrivelse} er nedenstående funktionelle krav opstillet. 
%  
%\begin{itemize}
%\item Systemet skal kunne oprette forbindelse til databasen
%	\\
%	\textit{Dette er nødvendigt for at gemme, hente og redigere data}
%
%\item Systemet skal kunne gemme data i databasen
%	\\
%	\textit{Dette er nødvendigt for at tilgå tidligere resultater}
%
%\item Systemet skal kunne hente data fra databasen
%	\\
%	\textit{Dette er nødvendigt for at tilgå tidligere resultater for både patienter og sundhedspersonale}	
%	
%\item Systemet skal kunne oprette nye brugere 
%	\\
%	\textit{Dette er nødvendigt for, at flere KOL-patienter kan anvende app'en}
%
%\item Systemet skal kunne redigere brugeroplysninger
%	\\
%	\textit{Dette er nødvendigt, hvis patienters tilstand ændres}
%
%\item Systemet skal have en login og logout funktion
%	\\
%	\textit{Dette er nødvendigt for at sikre patienters individuelle data}
%
%\item Systemet skal kunne kategorisere og vurdere  KOL-patienters helbredstilstand 
%	\\
%	\textit{Dette er nødvendigt for at kunne tilpasse træningssæt efter den enkelte KOL-patient}
%
%\item Systemet skal kunne sende notifikationer og give virtuelle belønninger 
%	\\
%	\textit{Dette er nødvendigt for at kunne motivere patienter til at udføre træning}
%
%\item Systemet skal kunne interagere med andre KOL-patienter
%	\\
%	\textit{Dette er nødvendigt for motivering mellem KOL-patienter samt skabe fællesskab}
%	\\
%	\textit{Dette er nødvendigt for at patienter kan tilgå hinandens virtuelle belønninger}
%	\\
%	\textit{Dette er nødvendigt for at kunne give notifikationer, hvis andre patienter træner}
%
%\end{itemize}
%
%\subsection{Non-funktionelle krav}
%De non-funktionelle krav er opstillet ud fra en den overbevisning, at dette ikke er krav til systemets funktionalitet, men stadig er relevant i relation til den bedste udbredelse, brugervenlighed og brugeroplevelse. 
%
%\begin{itemize}
%\item Systemet skal kunne fungere på en smartphone eller tablet 
%\item Systemet skal være brugervenligt
%	\\
%	\textit{Dette er nødvendigt, da KOL-patienter ofte er ældre}
%	
%\item Systemet skal tilgås med et medlemsnummer, navn og kodeord
%	\\
%	\textit{Dette er nødvendigt for at sikre, at KOL-patienter har deltaget i et rehabiliteringsforløb}
%	\\
%	\textit{Dette er nødvendigt for at adskille patienters data}
%\end{itemize}