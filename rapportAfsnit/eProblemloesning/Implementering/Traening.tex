\section{Tilpasning af træningsniveau}
I tilpasningscontrolleren beregnes den anbefalede træningstid. Dette gøre ud fra kategorisering, den daglige helbredstilstand og en tidligere evaluering, foretaget efter samme træningsform, -type og helbredstilstand.
Beregningen for den anbefalede træning er implementeret ved brug af if/else statement. Et udpluk af denne beregning fremgår af \autoref{fig:anbekode}.  
   
\begin{figure} [H]
\centering
\includegraphics[width=1\textwidth]{figures/imple/anbekode}
\caption{Udpluk af koden for anbefalet træning. If/else loops afgør, hvilket træningsniveau brugeren får anbefalet.}
\label{fig:anbekode}
\end{figure} 

\noindent
Af figuren ses beregningen af den anbefalede træningstid for brugeren med kategorisering \textit{A} og heldbredstilstanden på \textit{2}. Variationen for anbefalingen afhænger af evalueringen og kan i dette eksempel variere mellem \textit{15}, \textit{20} og \textit{25}  minutter. Metodekaldet \textit{setMinutter} referer til, at der sættes en bemærkning idet den anbefaldede træningstid er opnået.

********Her skal der komme noget træning efter ***********
