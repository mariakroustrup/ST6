\subsection{Unified Modellig Language}
En af de anvendte sprog indenfor objektorienteret programmering er standarden Unified Modelling Language (UML). Ud fra denne standard anvendes modeller til at visualisere struktur og egenskaber af systemet. Derudover relaterer metoderne til analyse og design af systemet. Modeller til at visulisere egenskaber er blandt andet use case diagrammer og aktivitetsdiagrammer. Til visualisering af struktur anvendes blandt andet klassediagrammer \cite{Fowler2004, Williams2004}.


\subsubsection{Use case diagrammer} 
Use case diagrammer benyttes til at illustrere aktørernes interaktion med et system samt, hvordan forskellige use cases interagerer mellem hinanden. Dertil er use case diagrammet med til at repræsentere funktionelle krav for systemet. \cite{Williams2004} Et eksempel på et use case diagram ses af \autoref{fig:use_case}.

\begin{figure} [H]
\centering
\includegraphics[width=0.5\textwidth]{figures/USE_CASE2}
\caption{Simpelt use case diagram.}
\label{fig:use_case}
\end{figure}

\noindent
Af \autoref{fig:use_case} ses aktørens interaktion med use case visualiseret som en streg mellem de to. I et use case diagram vil aktøren definere en person, der kan tilgå systemets funktionaliteter. Dette kan eksempelvis være en person, rolle, objekt eller en anden given genstand. Hertil vil den enkelte use case beskrive en handling eller funktionalitet i systemet.\cite{Fowler2004, Williams2004}


\subsubsection{Aktivitetsdiagrammer} 
Aktivitetsdiagrammer anvendes til at beskrive, hvad der sker i programmet, herunder proceduremæssig logik, business processer og arbejdsflow. Aktiviteter kan opdeles i subaktiviteter eller metoder. Subaktiviteter vil fremgå af diagrammet ved et rivesymbol, mens metoder vil fremgå ved syntaksen klasse-navn::metode-navn. Aktivitetsdiagrammer fortæller ikke, hvem der udfører aktiviteten, hertil kan der anvendes skillevægge, som viser, hvilken aktivitet en klasse eller organisation tilhører. For at holde et aktivitetsdiagram enkelt kan der anvendes et brillesymbol i en aktivitet. Denne aktivitet vil efterfølgende kunne beskrives yderligere i et nyt aktivitetsdiagram.\cite{Fowler2014} Symboler, der kan anvendes inden for aktivitetsdiagrammer, fremgår af \autoref{fig:aktivitetsdiagram}.

\begin{figure} [H]
\centering
\includegraphics[width=0.5\textwidth]{figures/aktivitetsdiagram}
\caption{Symboler der kan anvendes i aktivitetsdiagrammer \cite{Fowler2014}.}
\label{fig:aktivitetsdiagram}
\end{figure}


%Til at beskrive komplekse use cases eller klassemetoder anvendes aktivitetsdiagrammer. Dette giver overblik over flowet gennem de forskellige aktiviteter i den givne funktion eller metode. \cite{Fowler2004}    

%Såfremt der i et aktivitetsdiagram anvendes et 'brille' symbol, indikerer dette at den aktivitet i sig selv er kompleks og er beskrevet i et særskilt aktivitetsdiagram.  

\subsubsection{Klassediagrammer}
Klassediagrammer anvendes som redskab til at beskrive strukturen i et givent system og dermed skabe overblik over forskellige klasser og relationer, der indgår i systemet. 

Som det ses af \autoref{fig:klassediagram} identificeres hver klasse ud fra et unikt klassenavn, hvor der yderligere kan tildeles attributter og metoder til klassen.

\begin{figure} [H]
\centering
\includegraphics[width=0.5\textwidth]{figures/klassediag}
\caption{I klassediagrammer identificeres klasser ud fra et klassenavn, og dertilhørende attributter og metoder tilføjes nedenfor navnet.}
\label{fig:klassediagram}
\end{figure}

\noindent
Attributter og metoder kan markeres med symbolerne; +, - eller #, som symboliserer, at de henholdsvis er public, private eller protected, jf. \autoref{sec:oop}.

Relationerne mellem klasserne illustreres ved brug af forskellige pile, og disse kan navngives for at tydeliggøre forholdet mellem  klasserne. Yderligere kan multipliciteten angives ved at tilføje symbolet *, der angiver "mange", eller specifikke værdier i pilenes ender. 