\subsection{Software Development Lifecycle}
Software development lifecycle (SDLC) er en række trin, der kan følges, til udvikling af softwaresystemer. 




\subsection{Vandfaldsmodel}
Vandfaldsmodellen er den ældste og mest brugte SDLC model til udvikling af softwaresystemer. Modellen følger fem faser; analyse og kravspecifikationer, design, implementering, test samt vedligeholdelse. Disse faser gennemgås og dokumenteres enkeltvis førend næste fase påbegyndes. Dette bidrager til at sikre kvaliteten af systemudviklingen. Ved vandfaldsmodellen kan der dog også opstå problematikker i forhold til fejl. Disse fejl kan eksempelvis opstå i første fase, men først opdages i fjerde fase, hvorfor faserne derved skal gennemgås igen.\cite{Alshamrani2015,Bassil2012} Af \autoref{fig:vandfaldsmodel} fremgår vandfaldsmodellen med de fem faser. 

\begin{figure} [H]
\centering
\includegraphics[width=0.8\textwidth]{figures/vandfaldsmodel}
\caption{Vandfaldsmodellens fem faser bestående af analyse og kravspecifikation, design, implementering, test samt vedligeholdelse. Revideret \cite{Alshamrani2015,Bassil2012}.}
\label{fig:vandfaldsmodel}
\end{figure} 

\noindent
Den første fase, analyse og kravspecifikationer, er en omfattende analyse samt beskrivelse af systemets formål. Herunder opstilles funktionelle og non-funktionelle krav, der beskriver, hvilke funktioner samt begrænsninger systemet burde have. Under første fase udarbejdes ligeledes use-case diagrammer i sammenhæng med de funktionelle krav. Efter analyse og kravspecifikationer forekommer designet af systemet. I denne fase planlægges og designes en softwareløsning baseret ud fra første fases kravspecifikationer. Herunder udvælges blandt andet algoritme design, software arkitektur design, databasedesign samt definition af datastruktur. Tredje fase indebærer implementeringen. Denne fase har til formål at implementere og konvertere de opstillede kravspecifikationer samt design fra tidligere faser til et system. Denne konvertering foregår gennem programmering. Den fjerde fase omhandler test og kontrol af softwareløsningen i forhold til de opstillede kravspecifikationer. Vedligeholdelsesfasen, der er den sidste fase, indebærer eventuelle ændringer og forbedringer af softwaresystemet efter det er frigivet.\cite{Alshamrani2015,Bassil2012}








