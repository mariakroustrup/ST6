\section{Designklasser og sekvensdiagrammer}
I designklasser vælges der at opdele de klasser der blev defineret i \autoref{cha:analyse} til mindre klasser. Dette gøres for at specificere de definerede attributter og metoder. For hver controlklasse er der opstillet en eller flere boundaryklasser. 

I sekvensdiagrammer udarbejdes der diagrammer ud fra de opstillede designklasser. Dette gøres ud fra Model-View-Controller (MVC) arkitektur, der har ligheder med de definerede stereotyper. MVC anvendes til at organisere systemet og giver større fleksibilitet. 

\begin{itemize}
\item Model anvendes til at lagre data \cite{Brahma2015}.
\item View anvendes til interaktion mellem bruger og system \cite{Brahma2015}.
\item Controller anvendes til at kontrollere brugerinput og kalde metoder \cite{Brahma2015}.
\end{itemize}

\noindent
Beskrivelserne af designklasse og sekvensdiagrammer er opdelt i log ind, kategorisering, hovedmenu, tilpasning af træningsniveau, træning, resultater, venneliste og log ud. Derudover er der beskrevet database og entity i designklasser. 
