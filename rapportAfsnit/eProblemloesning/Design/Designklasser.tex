\section{Objektorienteret design}
Der vælges at opdele analyseklasserne, som tidligere er defineret, til mindre designklasser. Dette gøres for at specificere de definerede attributter og metoder, hvilket gør det lettere at modificere systemet. Ud fra de opstillede designklasser udarbejdes sekvensdiagrammer. I sekvensdiagrammerne er systemets klasser opdelt efter MVC. Objektet og livslinjer farvekodes afhængig af klassens MVC-type. Modelklasser er røde, viewklasser er blå, og controllerklasser er grønne. I sekvensdiagrammer findes desuden også database, som er illustreret med lilla. I sekvensdiagrammerne findes metoder, der henter data fra en klasse til en anden. Idet det ikke er en egentlig metode, der sender data, illustreres overførslen med en stiplet pil, når data sendes til en klasse efter en hent-metode er kaldt. 

Der er udarbejdet et samlet diagram over designklasser og relationerne i mellem, hvilket vil fremgå af bilag \ref{bilagA}. De enkelte designklasser og sekvensdiagrammer er opdelt i database, lagring af data, fejlmeddelelse, log ind, kategorisering, hovedmenu, tilpasning af træningsniveau, træning, resultater, venneliste, redigering af adgangskode og log ud. 
