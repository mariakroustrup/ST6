\subsection{Brugervenlighed}
Der er i \autoref{sec:funktionellekrav} opstillet et non-funktionelt krav om et brugervenligt system. For at opfylde dette krav, tages der udgangspunkt i gestaltlovene samt generelle egenskaber indenfor design af brugergrænseflader, der kan have indflydelse på brugervenligheden. 

Gestalt principperne bygger på en række love, der er opstillet af forskellige psykologer og anvendes til at designe visuelle elementer med henblik på forbedring af læring eller effektivisering af visuelle resultater. Der findes mange gestalt love. De mest anvendte til design af brugergrænseflader er; symmetri, regelmæssighed, lukkethed, prægnans\fxnote{Figur-grund eller prægnans omhandler det psykologiske fænomen, at man ikke kan opleve en figur uden samtidig at opleve dens baggrund}, fokuspunkt, ensartethed, nærhed, lighed og harmoni.\cite{Chang2002} Udover gestalt lovene har følgende egenskaber betydning for brugervenligheden \cite{ferre2001}:
\begin{itemize}
\item Systemets funktioner skal være lette at lære at anvende for uerfarne brugere
\item Systemet skal være effektivt at anvende i forhold til tiden, det tager at fuldføre en opgave
\item Det skal være let at huske, hvordan systemet anvendes efter længere perioder uden brug af systemet
\item Fejlraten ved brug af systemet skal være så lav som muligt
\item Systemet skal være tilfredsstillende for brugeren at anvende
\end{itemize}
 
 
