\subsection{Venneliste}
Venneliste inddeles i en grænseflade og dertilhørende controller, som det fremgår af \autoref{fig:MVCVenneliste}. 

\begin{figure} [H]
\centering
\includegraphics[width=0.9\textwidth]{figures/MVC/Venneliste}
\caption{Designklasser for venneliste.}
\label{fig:MVCVenneliste}
\end{figure}

\noindent
\textit{VennelisteGrænsefladen} indeholder tekstfelter for fornavn og efternavn på de brugere de følger. Derudover er der opstillet tekstfelt for medlemsID, hvor brugeren kan angive medlemsID'et på en bruger de ønsker at følge. Dertil er der opstillet en tilhørende søge knap, af typen button, der ved tryk indikere at brugeren har angivet medlemsID på brugeren den ønsker at følge. Derudover har brugeren mulighed for at vælge en bruger eller gå tilbage via en vælg eller tilbage  knap, der også er af typen button. 


Der er til \textit{VennelisteGrænsefladen} opstillet en \textit{VennelisteController}, der har til formål at vise en oversigt over vennelisten når grænsefladen for vennelisten tilgås. Brugeren har via vælg knappen mulighed for at tilgå en bruger den følger
Controlleren lytter på om brugeren trykker på vælg knappen eller søg knappen. Vælg knappen giver brugeren mulighed for at tilgå enkelt bruger ud fra vennelisten. Trykker brugeren på søg knappen kontrollere controlleren om det angivne medlemsID findes i databasen, herefter kan brugeren trykke på vælg. Trykkes der på en af vælg knapperne, enten i vennelisten eller efter søgning på medlemsID vises \textit{VenGrænsefladen}, som fremgår af \autoref{MVCVen}. Controlleren lytter på tilbage knappen, hvis denne trykkes på vises forrige grænseflade. 


\subsubsection{Ven}
Ven inddeles i en grænseflade og tilhørende controller, som det fremgår af \autoref{fig:MVCVen}.

\begin{figure} [H]
\centering
\includegraphics[width=0.9\textwidth]{figures/MVC/Ven}
\caption{Designklasser for ven.}
\label{fig:MVCVen}
\end{figure}

\noindent
I \textit{VenGrænseflade} vises tekstfelter for valgt bruger, herunder MedlemsID, fornavn og efternavn. Dertil er der opstillet en følg knap og en tilbage knap af typen button. Følg knap er kun tilgængelig, hvis brugeren ikke følger den valgte ven. Derudover vises en tabel med den valgte brugers belønninger.


Til \textit{VenGrænsefladen} er der opstillet en \textit{VenController}, som har til formål vise brugeroplysninger, herunder MedlemsID, fornavn og efternavn samt belønningstabel for den valgte bruger. Controlleren lytter på om brugeren trykker på følg knappen, hvis brugeren ikke allerede følger eller trykker på tilbage knappen. Trykkes der på følg knappen sendes MedlemsID'et på den bruger der ønskes at følges til databasen, hvorefter vennerelation gemmes i denne. Vælger brugeren at trykke på tilbage knappen vises den forrige grænseflade.   