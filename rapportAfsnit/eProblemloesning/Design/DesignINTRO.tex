I dette kapitel beskrives design af app'en samt design af database. I design af app'en tages der udgangspunkt i de analyseklasser, der blev identificeret i \autoref{cha:analyse}, hvoraf disse omdannes til designklasser. Herefter visualiseres sammenspillet mellem de forskellige designklasser i sekvensdiagrammer. Design af database, der vil fremgå til sidst af kapitlet, vil visualiseres i et ER-diagram samt tilhørende schema. 

\section{Designafgrænsning} \label{sec:brugervenlighed}
Systemet designes med henblik på at opfylde kravsspecifikationerne, der blev udarbejdet i systemanalysen jf. \autoref{sec:funktionellekrav}. I forbindelse med de opstillede funktionelle krav foretages der afgrænsninger for at gøre design og efterfølgende implementering af systemet mere konkret. Der afgrænses til konditionstræning, da app'en formål ikke er at udforme et træningssæt til KOL-patienter. Dog forventes det, at implementeringen af styrketræning og vejrtrækningsøvelser er lignende. 

Der er opstillet et non-funktionelt krav om et brugervenligt system. For at opfylde dette krav, tages der udgangspunkt i gestaltlovene samt generelle egenskaber indenfor design af brugergrænseflader, der kan have indflydelse på brugervenligheden. 

Gestalt principperne bygger på en række love, der er opstillet af forskellige psykologer og anvendes til at designe visuelle elementer med henblik på forbedring af læring eller effektivisering af visuelle resultater. Der findes mange gestalt love. De mest anvendte til design af grænseflader er; symmetri, regelmæssighed, lukkethed, prægnans, fokuspunkt, ensartethed, nærhed, lighed og harmoni.\cite{Chang2002} Udover gestaltlovene har følgende egenskaber betydning for brugervenligheden \cite{ferre2001}:
\begin{itemize}
\item Systemets funktioner skal være lette at lære at anvende for uerfarne brugere.
\item Systemet skal være effektivt at anvende i forhold til tiden, det tager at fuldføre en opgave.
\item Det skal være let at huske, hvordan systemet anvendes efter længere perioder uden brug af systemet.
\item Fejlraten ved brug af systemet skal være så lav som muligt.
\item Systemet skal være tilfredsstillende for brugeren at anvende.
\end{itemize}
 

 





