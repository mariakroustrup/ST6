\section{Design af database}
I forhold til sikkerhedsmæssige årsager ønskes det, at data ikke er gemt på app’en, men er lagret i en database, der kan hentes via app’en. KOL-patienter oprettes i databasen af sundhedspersonalet i forbindelse med rehabiliteringsforløb, som nævnt i \autoref{sec:registrering}. Databasen skal indeholde oplysninger om de enkelte KOL-patienter, herunder resultater opnået ved træning og vennerelation. 

\subsection{ER-diagram}
Modellering af databasen udarbejdes ud fra et ER-diagram. ER-diagrammet relaterer sig til én enkelt KOL-patient i databasen. Databasen tager udgangspunkt i entiteter, som sundhedspersonale, KOL-patient, vennerelation, træning, konditionstræning, styrketræning og vejrtrækningsøvelser. ER-diagrammet for databasen fremgår af \autoref{fig:ERdiagram}.


\begin{figure} [H]
\centering
\includegraphics[width=1\textwidth]{figures/Aktivitetsdiagram/ERdiagram}
\caption{ER-diagram for database}
\label{fig:ERdiagram}
\end{figure} 

\noindent
Af \autoref{fig:ERdiagram} ses ER-diagrammet over databasen, hvori KOL-patienter oprettes og informationer om patienterne lagres. Sundhedspersonalet fremgår som en stærk entitet, der opretter alle KOL-patienter, hvor én KOL-patient kan kun oprettes i databasen én gang. Sundhedspersonale registreres med et \textit{personaleID}, fornavn, efternavn og afdelingsID, hvor \textit{personaleID} er primærnøglen, der kan identificere personen. Den enkelte KOL-patient er en stærk entitiet, som registreres med primærnøglen, \textit{medlemsID}, samt, brugernavn, fornavn, efternavn, adgangskode og kategorisering. Derudover fremgår de svage entiteter, herunder vennerelation, træning, konditionstræning, styrketræning og vejrtrækningsøvesler. Én KOL-patienter har mange vennerelationer, som kan identificeres ved \textit{venMedlemsID} og \textit{medlemsID}. Det samme gør sig gældende for træninger, hvor én KOL-patient kan udføre mange træninger, der identificeres ved  \textit{dato/tid} og \textit{medlemsID}. Af \autoref{fig:ERdiagram} fremgår det at konditionstræning, styrketræning og vejrtrækningsøvelser nedarver flere attributter fra træningen, da entiteterne har ligheder og forskelligheder. Kondition har foruden de nedarvede attributter afstand. 


\subsection{Schema}
ER-diagrammet omskrives til schema for at kunne normalisere og implementere databasen. Normaliseringen anvendes med henblik på at reducere redundans og inkonsistens. Schema er i anden normalform og fremgår af \autoref{tab:schema}. 

\begin{table}[H]
\centering
\begin{tabular}{|l|l|}
\hline
\textbf{Stærke entiteter} & \begin{tabular}[c]{@{}l@{}}Sundhedspersonale = (\underline{personaleID}, afdelingsID, personaleFornavn, \\ personaleEfternavn)\\
KOL-patient = ( \underline{medlemsID}, brugernavn, adgangskode, fornavn, efternavn, \\ kategorisering)\end{tabular}                     \\ \hline
\textbf{Svage entiteter}  & \begin{tabular}[c]{@{}l@{}}Træning = (\underline{medlemsID}, \underline{tid/dato}, type, nedtælling, daglig helbredstilstand, \\ evaluering, måleenheder, timer, belønninger)\\ Vennerelation = (\underline{medlemsID}, \underline{venMedlemsID}, venBrugernavn, venBelønninger)\end{tabular} \\ \hline
\end{tabular}
\caption{ER-diagram for databasen omskrevet til schema på anden normalform.}
\label{tab:schema}
\end{table}

\noindent
Schemaet på anden normalform optimeres til tredje normalform, da det giver bedre muligheder ved implementering af databasen. For at komme på tredje normalform fjernes dimensioner, der ikke har en direkte tilgang til primærnøglen. Tredje normalform ses af \autoref{tab:schema3}.

\begin{table}[H]
\centering
\begin{tabular}{|l|l|}
\hline
\textbf{Stærke entiteter} & \begin{tabular}[c]{@{}l@{}}Sundhedspersonale = (\underline{personaleID}, afdelingsID,)\\
KOL-patient = ( \underline{medlemsID}, adgangskode, kategorisering)\end{tabular}                     \\ \hline
\textbf{Svage entiteter}  & \begin{tabular}[c]{@{}l@{}}Træning = (\underline{medlemsID}, \underline{tid/dato}, type, nedtælling, daglig helbredstilstand, \\ evaluering, måleenheder, timer, belønninger)
\\ Vennerelation = (\underline{medlemsID}, \underline{venMedlemsID}, venBelønninger)\end{tabular} \\ \hline
\end{tabular}
\caption{ER-diagram for databasen omskrevet til schema på tredje normalform.}
\label{tab:schema3}
\end{table}




