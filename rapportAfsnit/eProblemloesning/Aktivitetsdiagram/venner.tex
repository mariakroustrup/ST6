\subsection{Social relationer} 
For at motivere KOL-patienter til en regelmæssig træning, vælges at integrere muligheden for sociale relationer i app'en. Dette muliggøre, at brugere kan følge hinanden og derved se, hvilke belønninger andre brugere har opnået. Hertil skal det være muligt for brugeren at tilføje og fjerne venner fra deres venneliste. 
Af \autoref{fig:venner} fremgår et aktivitetsdiagram for sociale relationer. 

\begin{figure} [H]
\centering
\includegraphics[width=0.65\textwidth]{figures/aktivitetsdiagram/venner}
\caption{Aktivitetsdiagram for sociale relationer.}
\label{fig:venner}
\end{figure}

\noindent
Vennelisten viser en oversigt over andre brugere, som den individuelle bruger vælger at følge. Heraf skal det være muligt for den individuelle bruger at tilgå information fra disse andre brugere. Denne information begrænses til at vise vundet belønninger for de andre brugere, da den resterende information anses som værende personlig til den enkelte bruger. 
Fra vennelisten skal den individuelle bruger have mulighed for at fjerne brugere fra vennelisten eller tilføje andre brugere, hvortil disse funktioner uddybes af henholdsvis \autoref{fig:venner} og \autoref{fig:tilfoejven}.  
Ønskes det at fjerne en bruger fra vennelisten, sendes informationer om den valgte bruger til databasen, hvorved vennerelationen slettes. Derefter returneres en bekræftigelse, der viser, at den valgte bruger er fjernet fra vennelisten. 

\begin{figure} [H]
\centering
\includegraphics[width=0.65\textwidth]{figures/aktivitetsdiagram/foelgnyven}
\caption{Aktivitetsdiagram for tilføjelse af venner til vennelisten.}
\label{fig:tilfoejven}
\end{figure}

Ved tilføjelse af nye bruger indtastes brugernavnet eller medlemsID'et på den givne bruger. Brugernavnet eller medlemsID'et sendes til databasen, der returnerer et brugernavn, medlemsID, vennerelation eller fejlmelding til systemet. Hvis der allerede er en relation mellem de to brugere eller, at brugerinformationen ikke eksisterer i databasen, forekommer en fejlmelding, hvortil et andet brugernavn eller medlemsID kan indtastes på ny. 
Såfremt, at der ikke forekommer en fejlmelding vises den søgte bruger, hvortil det er muligt at følge vedkommende. 
Vælges dette, sendes en anmodning til databasen som opretter en vennerelation mellem de to brugere. Efterfølgende returneres en bekræftelse om vennerelationen til systemet.