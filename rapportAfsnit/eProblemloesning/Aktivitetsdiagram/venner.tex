\subsection*{Venneliste} 
For at motivere brugere til regelmæssig træning, vælges at integrere muligheden for sociale relationer i app'en. Dette muliggør, at brugere kan følge hinanden og derved se, hvilke belønninger andre brugere har opnået. Hertil skal det være muligt for brugeren at tilføje samt fjerne brugere fra vennelisten. 
Af \autoref{fig:venner} fremgår et aktivitetsdiagram for vennelisten.

\begin{figure} [H]
\centering
\textbf{Aktivitetsdiagram: Venneliste}\par\medskip
\includegraphics[width=0.9\textwidth]{figures/aktivitetsdiagram/venner}
\caption{Aktivitetsdiagram for venneliste.}
\label{fig:venner}
\end{figure}

\noindent
Tilgår brugeren sin venneliste fra hovedmenuen, hentes brugerens venner fra databasen, hvorefter de opstilles i en liste. Systemet viser hertil en grænseflade for vennelisten, der indeholder en oversigt over de brugere, som den individuelle bruger følger. Brugeren kan vælge at søge efter en ny bruger eller vælge en bruger fra vennelisten. Ønsker brugeren at søge efter en ny, skal brugeren angive medlemsID på denne. Systemet sender det indtastede medlemsID til databasen, som validerer om brugeren findes i databasen. Findes brugeren ikke i databasen, sendes en fejlmeddelelse, og brugeren har igen mulighed for at angive medlemsID på en anden bruger fra grænsefladen for venneliste. Er brugeren i databasen, hentes brugerens oplysninger, herunder medlemsID og navn, samt brugerens belønninger. Vælger brugeren en ven fra vennelisten, hentes ligeledes brugerens oplysninger samt belønninger. Når brugerens data er hentet fra databasen, vises grænsefladen for brugers belønninger. Det er hertil muligt at følge brugeren, hvis vedkommende ikke allerede følges samt fjerne brugeren, hvis det ønskes ikke at følge brugeren mere. Ønskes det at følge eller fjerne en bruger, sender systemet en anmodning, hvorefter en vennerelation oprettes eller slettes i databasen og en bekræftelse sendes. 


%Heraf skal det være muligt for den individuelle bruger at tilgå information fra disse. Denne information begrænses til at vise vundne belønninger for de andre brugere, da den resterende information anses som værende personlig til den enkelte bruger. 
%Fra vennelisten skal den individuelle bruger have mulighed for tilføje andre brugere, hvilket fremgår af \autoref{fig:tilfoejven}.  
% Ønskes det at fjerne en bruger fra vennelisten, sendes informationer om den valgte bruger til databasen, hvorved vennerelationen slettes. Derefter returneres en bekræftelse, der viser, at den valgte bruger er fjernet fra vennelisten. 

%\begin{figure} [H]
%\centering
%\textbf{Aktivitetsdiagram: Opret vennerelation}\par\medskip
%\includegraphics[width=0.8\textwidth]{figures/aktivitetsdiagram/foelgnyven}
%\caption{Aktivitetsdiagram for tilføjelse af venner til vennelisten.}
%\label{fig:tilfoejven}
%\end{figure}

%\noindent
%Ønskes det at oprette vennerelation, indtastes fornavn og efternavn på den givne bruger. Denne information sendes til databasen, der returnerer fornavn og efternavn på fundne brugere i databasen, eller fejlmeddelelse til systemet. Hvis der allerede er en relation mellem de to brugere eller, at brugerinformationen ikke eksisterer i databasen, forekommer en fejlmelding, hvortil en ny brugersøgning kan indtastes på ny. 
%Såfremt, at der ikke forekommer en fejlmeddelelse vises den søgte bruger, hvortil det er muligt at oprette vennerelation. 
%Vælges dette, sendes en anmodning til databasen, som gemmer en vennerelation mellem de to brugere. Efterfølgende returneres en bekræftelse om vennerelationen til systemet.