\section{Funktionalitet}
I dette afsnit beskrives funktionaliteterne, der er udarbejdet ud fra systembeskrivelsen samt use case diagrammet. De enkelte funktionaliteter er opdelt efter registrering, log ind, redigering, kategorisering af KOL, daglig helbredstilstand, træning, resultater og sociale relationer. 

\subsection*{Registrering} \label{sec:registrering}
Inden KOL-patienter kan anvende app'en skal de registreres som brugere af systemet. Dette skal foregå i forbindelse med rehabiliteringsforløbet, hvor sundhedspersonale opretter patienterne i databasen. Patienterne får tilknyttet et medlemsID og en adgangskode. Dette er vigtigt for, at reducere risikoen for misbrug af personlige oplysninger, da uvedkommende ellers kan have mulighed for at tilgå informationerne via netadgang eller enheden \cite{Sundhedsdatastyrelsen2016}. MedlemsID'et skal bestå af tal, eksempelvis \textit{01170301}, som er sammensat ud fra lokalisation, årstal og måned for påbegyndt rehabilieringsforløb samt nummerering af den enkelte KOL-patient.
Adgangskoden, der bliver udleveret af sundhedspersonalet, kan senere ændres i app'en, hvis en personlig adgangskode ønskes. 

Under registrering skal KOL-patienter ligeledes vælge et brugernavn, som gør dem identificerbare, således andre brugere kan følge dem. Brugernavn tilføjes til databasen, og KOL-patienter kan dermed vælge at logge ind på app'en ved brug af brugernavn eller  medlemsID samt adgangskode. 

I forbindelse med registrering skal sundhedspersonalet introducere KOL-patienter til brugen af app'en. Herunder skal de hjælpe KOL-patienterne med at kategorisere patientens sygdom før app'en anvendes til træning i hjemmet. Dette skal gøres i et forsøg på at skabe tryghed hos patienterne, da denne kategorisering har betydning for, hvilket træningsniveau patienten senere får foreslået af app'en. Der er desuden mulighed for at kunne få besvaret eventuelle tvivlsspørgsmål, der kan opstå første gang app'en anvendes. 


