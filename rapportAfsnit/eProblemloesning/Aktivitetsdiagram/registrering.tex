\section{Funktionalitet}
I dette afsnit beskrives funktionaliteterne, der er udarbejdet ud fra systembeskrivelsen samt use case diagrammet. De enkelte funktionaliteter er opdelt efter registrering, log ind, kategorisering af KOL, tilpasning af træningsniveau, træning, resultater, venneliste, redigering af adgangskode og log ud. Nogle af funktionaliteterne er beskrevet i aktivitetsdiagrammer, som er opdelt i bruger, systemet og database. Der er i nogle af aktivitetsdiagrammerne angivet et billesymbol, hvilket betyder at aktiviten vil uddybes i et andet aktivitetsdiagram. Når et aktivitetsdiagram starter antages det, at brugeren har trykket på den gældende aktivitet. Efter hver endt aktivitet eller ved at trykke tilbage vises en hovedmenu. 

\subsection*{Registrering} \label{sec:registrering}
Inden KOL-patienter kan anvende app'en skal de registreres som brugere af systemet. Dette skal foregå i forbindelse med rehabiliteringsforløbet, hvor sundhedspersonalet opretter patienterne i databasen. Patienterne får tilknyttet et medlemsID og en randomiseret adgangskode. Dette er vigtigt for at reducere risikoen for misbrug af personlige oplysninger, da uvedkommende kan have mulighed for at tilgå informationerne via netadgang eller enheden \cite{Sundhedsdatastyrelsen2016}. MedlemsID'et skal bestå af tal, eksempelvis \textit{11170301}, som er sammensat ud fra lokalisation, årstal og måned for påbegyndt rehabilieringsforløb samt nummerering af den enkelte KOL-patient.

Under registrering oprettes KOL-patienter ligeledes med fornavn og efternavn, der skal gøre dem identificerbare, således andre brugere kan følge dem. 

I forbindelse med registrering skal KOL-patienterne logge ind, hvortil sundhedspersonalet introducerer KOL-patienter til brugen af app'en. Herunder skal de hjælpe KOL-patienterne med at kategorisere patientens sygdom før app'en anvendes til træning i hjemmet. Dette skal gøres i et forsøg på at skabe tryghed hos patienterne, da denne kategorisering har betydning for, hvilket træningsniveau patienten senere får foreslået af app'en. Der er desuden mulighed for at kunne få besvaret eventuelle tvivlsspørgsmål, der kan opstå første gang app'en anvendes. 


