\section{Analyseklasser} \label{sec:analyseklasser}
For at analyseklasserne kan udarbejdes, foretages en analyse ud fra systembeskrivelsen, use case og funktionaliteter til at identificere substantiver og verber. Dette gøres for at sikre, at alle funktionaliteter indgår i design af klassediagrammer. Substantiver og verber fra analysen fremgår af \autoref{tab:subverb}.

\begin{table}[H]
\centering
\includegraphics[width=0.8\textwidth]{figures/aktivitetsdiagram/substantiveverber}
\caption{Substantiver og verber identificeret ved analyse af systembeskrivelse, use case samt funktionaliteter.}
\label{tab:subverb}
\end{table}

\noindent
De fremhævede substantiver, \textit{brugeroplysninger}, \textit{tilpasning af træningsniveau}, \textit{træning}, \textit{resultater}, \textit{venneliste} og \textit{database}, identificeres som klasser. Under hver klasse fremgår deres tilhørende attributter, der beskriver den overordnede klasse. Verberne betegner de metoder, der kan tilgås i de forskellige klasser. 

Efter substantiver og verber er identificeret inddeles disse i analyseklasser og opdeles i typerne, entity og control. Dette kan ses af \autoref{fig:analyseklasse}. 

\begin{figure}[H]
\centering
\includegraphics[width=0.8\textwidth]{figures/aktivitetsdiagram/analyseklasser}
\caption{Analyseklasser udarbejdet ud fra de identificerede substantiver og verber.}
\label{fig:analyseklasse}
\end{figure}

\noindent
Af \autoref{fig:analyseklasse} fremgår relationen mellem klasserne og deres tilhørende attributter samt metoder. Controlklasserne indeholder metoderne, der skal udføres når denne tilgås, hvorimod entityklasserne har til formål at lagre data. \textit{Brugeroplysninger} og \textit{Træning} er defineret som entityklasser med tilhørende controlklasser. Entityklassen \textit{Træning} har yderligere en relation til controlklassen \textit{Tilpasning af træningsniveau}, da informationer fra denne ligeledes gemmes i denne entity. Derudover er opstillet tre controlklasser, herunder \textit{Resultater}, \textit{Venneliste} og \textit{Database}. Klassen, \textit{Database}, tilgås fra samtligt controlklasser, da formålet med denne klasse er at kommunikere med databasen. 