\subsection*{Resultater}
Fra app'ens hovedmenu kan brugeren tilgå sine resultater, der visualiseres grafisk og ved virtuelle belønninger.
Aktivitetsdiagrammet over resultater fremgår af \autoref{fig:resultater}.

\begin{figure} [H]
\centering
\textbf{Aktivitetsdiagram: Resultater}\par\medskip
\includegraphics[width=0.95\textwidth]{figures/aktivitetsdiagram/Resultater}
\caption{Aktivitetsdiagram over resultater.}
\label{fig:resultater}
\end{figure}

\noindent
Under resultater er det muligt for brugeren at se sin ugentlige træning samt virtuelle belønninger. Idet brugeren tilgår resultater fra hovedmenuen, hentes resultater, der vises som en grafisk udvikling, fra databasen. Brugeren har fra grænsefladen for grafisk udvikling mulighed for at tilgå sine belønninger. Ønskes dette, hentes brugerens belønninger fra databasen, hvorefter de visualiseres i grænsefladen for belønninger. Belønningerne varierer afhængig af træningsform, og der kan opnås forskellige belønninger inden for forskellige kategorier. 

%Et eksempel på fordeling af belønninger i forskellige kategorier fremgår af \autoref{tab:beloenninger}.
%
%\begin{table} [H]
%\centering
%\includegraphics[width=1\textwidth]{figures/aktivitetsdiagram/beloenninger}
%\caption{Eksempel på belønninger opnået ved træning inden for forskellige kategorier.}
%\label{tab:beloenninger}
%\end{table}
%
%\noindent
%Ud fra \autoref{tab:beloenninger} fremgår et eksempel på fordeling af virtuelle belønninger, der er opdelt efter afstand, tid og antallet af gennemførte træninger. 