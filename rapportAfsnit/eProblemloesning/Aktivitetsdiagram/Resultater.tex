\subsection*{Resultater}
Fra app'ens hovedmenu kan brugeren tilgå sine resultater. På denne måde er det muligt for brugeren at få et overblik over udviklingen samt udførte træninger.
Aktivitetsdiagrammet over resultater fremgår af \autoref{fig:resultater}.

\begin{figure} [H]
\centering
\textbf{Aktivitetsdiagram: Resultater}\par\medskip
\includegraphics[width=0.9\textwidth]{figures/aktivitetsdiagram/Resultater}
\caption{Aktivitetsdiagram over resultater.}
\label{fig:resultater}
\end{figure}

\noindent
Under resultater er det muligt for brugeren at følge sine ugentlige træningsresultater. Derfra har brugeren mulighed for at vælge daglige træningsresultater og tilgå belønninger. Belønningerne varierer afhængig af træningsform og der kan opnås forskellige belønninger inden for forskellige kategorier. Et eksempel på fordeling af belønninger i forskellige kategorier fremgår af \autoref{tab:beloenninger}.

\begin{table} [H]
\centering
\includegraphics[width=1\textwidth]{figures/aktivitetsdiagram/beloeninnger}
\caption{Eksempel på belønninger opnået ved træning inden for forskellige kategorier.}
\label{tab:beloenninger}
\end{table}

\noindent
Ud fra \autoref{tab:beloenninger} fremgår et eksempel på fordeling af virtuelle belønninger, der er opdelt efter afstand, tid og antallet af gennemførte træninger. 