I dette kapitel beskrives implementeringen af app'en, der omhandler omsætningen af design til kode. Det er valgt at implementere app'en i Android Studio version 2.3 og programmere i Java, da dette er et objektorienteret programmeringssprog. De forskellige controllerer og views, der blev defineret i sekvensdiagrammerne implementeres i henholdsvis Java-klasser og XML-filer. Database samt modeller er implementeret i MySQL ved brug af phpMyAdmin, der er et online databaseadministrationssystem og understøtter Structured Query Language (SQL) \fxnote{Her skal muligvis tilføjes noget}....

I \autoref{cha:design} er analyse- og designklasser samt funktionsnavne navngivet på dansk, hvorfor bogstaverne æ, ø og å forekommer. Disse symboler anvendes ikke under implementeringen for at undgå fejl.