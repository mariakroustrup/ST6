\section{Systembeskrivelse} \label{sec:systembeskrivelse}
I dette projekt udvikles en app, der har til formål at kunne anvendes af KOL-patienter som et redskab til opretholdelse af de gavnlige effekter, der opnås under rehabiliteringsforløb.

Under rehabiliteringsforløb lærer KOL-patienter forskellige træningsøvelser, som har en symptomlindrende effekt, jævnfør afsnit \ref{sec:rehabilitering}. App’en, der udvikles, skal foreslå træningsøvelser, som KOL-patienterne kan udføre, så de fortsat får udført fysisk aktivitet i hjemmet efter endt rehabiliteringsforløb. Disse træningsøvelser skal udvælges på baggrund de øvelser, der udføres under rehabiliteringsforløb, således at KOL-patienterne allerede har erfaringer og kendskab til de øvelser, som app’en kan foreslå.
Der er forskel på, hvor meget fysisk aktivitet forskellige KOL-patienter kan udføre, og der er derved forskel i varighed og intensitet af de træningsøvelser, som patienterne kan holde til. For at tage højde for dette under anbefaling af træningsøvelser, udvælges øvelserne på baggrund af den enkelte patients sværhedsgrad af KOL. Sværhedsgraden bestemmes ud fra ABCD-kategoriseringen, jævnfør \ref{sec:klassifikation}, som patienten skal angive ved oprettelse af brugerkonto til app’en.
Der kan opleves dag-til-dag variationer i KOL-patienters symptomer, hvilket medfører, at mængden af fysisk aktivitet, de kan holde til, dagligt kan variere. Som nævnt i afsnit \ref{sec:rehabilitering} kan KOL-patienter ved udførelse af fysisk aktivitet desuden opleve angst som følge af åndenød. For at reducere risikoen for, at KOL-patienter får negative oplevelser ved anvendelse af app’en, hvilket eksempelvis ville kunne ske, hvis der foreslås træningsøvelser på for højt niveau i forhold til patientens tilstand, skal app’en kunne tage højde for de daglige variationer i patientens sygdom. Dette imødekommes ved, at app’en spørger patienten om, hvordan symptomerne opleves den pågældende dag. Ud fra dette tilpasses træningsøvelserne, som KOL-patienterne bliver foreslået. Efter en udført træningssession skal patienten evaluere træningen i forhold til sværhedsgraden af træningen, så denne evaluering kan medtages i udvælgelsen af den efterfølgende dags anbefalede træning. (skal der være mere om dette?) Herved kan træningen tilpasses den enkelte KOL-patients tilstand.

For at hjælpe KOL-patienterne med vedligeholdelse af den daglige træning, skal app’en virke motiverende for patienterne. Dette gøres blandt andet ved at gøre det muligt for KOL-patienten at følge sin egen udvikling via app’en. App’en skal desuden gøre KOL-patienten opmærksom på, at det er lang tid siden sidste træning, hvis dette er tilfældet. For at øge motivationen hos patienten skal KOL-patienten derudover kunne opnå belønninger ved at udføre træningssessioner.
Som nævnt i afsnit \ref{sec:efterRehabilitering} er det sociale fællesskab en væsentlig faktor i opretholdelse af resultaterne fra rehabiliteringsforløb. Ved at indføre denne faktor i app’en kan dette være med til at motivere KOL-patienten til vedligeholdelse af den forbedrede livsstil. Dette gør det desuden muligt, at gøre patienten opmærksom på, at andre brugere af app’en har udført en træningssession, hvilket kan virke motiverende.
Sundhedsfagligt personale skal kunne tilgå KOL-patienternes resultater, så de kan følge med i udviklingen. De har herved mulighed for at informere patienterne om, hvorvidt de træner for lidt eller om de gør et godt arbejde, hvilket også kan have en motiverende effekt på KOL-patienterne.

\subsubsection{Funktionelle krav}
Ud fra informationer fundet gennem problemanalysen i \autoref{cha:problem} og beskrivelsen af systemet i \autoref{sec:systembeskrivelse} er nedenstående funktionelle krav blevet opstillet. 
  
\begin{itemize}
\item Skal kunne oprette forbindelse til databasen
	\\
	\textit{Nødvendig for at gemme, hente og redigere data}

\item Skal kunne gemme data i databasen
	\\
	\textit{Nødvendigt for at se tidligere resultater}

\item Skal kunne hente data fra databasen
	\\
	\textit{Nødvendigt for at se tidligere resultater for både patienter og sundhedspersonale (+ andre KOL-patienter)}	
	
\item Skal kunne oprette nye brugere 
	\\
	\textit{Nødvendigt for at flere KOL-patienter kan anvende denne}

\item Skal kunne redigere brugeroplysninger
	\\
	\textit{Nødvendigt, hvis patientens tilstand forværres eller forbedres}

\item Skal have log ind og log ud funktion
	\\
	\textit{Nødvendigt for at kunne anvende appen}

\item Skal kunne kategorisere og vurdere helbredstilstand KOL-patienter
	\\
	\textit{Nødvendig for at kunne tilpasse træningssæt efter den enkelte KOL-patient}

\item Skal kunne sende notifikationer og give belønninger 
	\\
	\textit{Nødvendig for at kunne motivere patienter til at udføre træning}

\item Skal kunne interagere med andre KOL-patienter
	\\
	\textit{Nødvendig for motivering mellem KOL-patienter samt skabe fællesskab}

\end{itemize}

\subsubsection{Non-funktionelle krav}
De non-funktionelle krav er opstillet ud fra en den overbevisning at dette ikke er krav til systemets funktionalitet, men stadig er relevant i relation til den bedste udbredelse, brugervenlighed og brugeroplevelse. 

\begin{itemize}
\item Skal fungere på en smartphone eller tablet 
\item Det skal være brugervenligt
	\\
	\textit{Patienter med KOL er ofte ældre}
	\\ 
	\textit{Designet skal være ens i gennem hele appen} 
	
\item Skal tilmeldes med et medlemsnummer, navn og kodeord
\end{itemize}