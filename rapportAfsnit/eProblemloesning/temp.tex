\subsubsection{Funktionelle krav}
Ud fra informationer fundet gennem problemanalysen i \autoref{cha:problem} og beskrivelsen af systemet i \autoref{sec:systembeskrivelse} er nedenstående funktionelle krav blevet opstillet. 
  
\begin{itemize}
\item Skal kunne oprette forbindelse til databasen
	\\
	\textit{Nødvendig for at gemme, hente og redigere data}

\item Skal kunne gemme data i databasen
	\\
	\textit{Nødvendigt for at se tidligere resultater}

\item Skal kunne hente data fra databasen
	\\
	\textit{Nødvendigt for at se tidligere resultater for både patienter og sundhedspersonale (+ andre KOL-patienter)}	
	
\item Skal kunne oprette nye brugere 
	\\
	\textit{Nødvendigt for at flere KOL-patienter kan anvende denne}

\item Skal kunne redigere brugeroplysninger
	\\
	\textit{Nødvendigt, hvis patientens tilstand forværres eller forbedres}

\item Skal have log ind og log ud funktion
	\\
	\textit{Nødvendigt for at kunne anvende appen}

\item Skal kunne kategorisere og vurdere helbredstilstand KOL-patienter
	\\
	\textit{Nødvendig for at kunne tilpasse træningssæt efter den enkelte KOL-patient}

\item Skal kunne sende notifikationer og give belønninger 
	\\
	\textit{Nødvendig for at kunne motivere patienter til at udføre træning}

\item Skal kunne interagere med andre KOL-patienter
	\\
	\textit{Nødvendig for motivering mellem KOL-patienter samt skabe fællesskab}

\end{itemize}

\subsubsection{Non-funktionelle krav}
De non-funktionelle krav er opstillet ud fra en den overbevisning at dette ikke er krav til systemets funktionalitet, men stadig er relevant i relation til den bedste udbredelse, brugervenlighed og brugeroplevelse. 

\begin{itemize}
\item Skal fungere på en smartphone eller tablet 
\item Det skal være brugervenligt
	\\
	\textit{Patienter med KOL er ofte ældre}
	\\ 
	\textit{Designet skal være ens i gennem hele appen} 
	
\item Skal tilmeldes med et medlemsnummer, navn og kodeord
\end{itemize}