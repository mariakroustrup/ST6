\chapter{Metode}
\section{Objektorienteret programmering}
Objektorienteret programmering(OOP) er et programmeringsparadigme, som anvendes til at analysere, designe, implementere samt udvikle app's. I OOP opdeles koden i objekter. Objekter repræsenterer hver sin del af app'en og indeholder sin egen data og logik. Hvert objekt er en instans af en bestemt klasse. En klasse opbygges af en eller flere instanser og fungerer som en opskrift. Klasser er defineret ud fra egenskaber og metoder. En egenskab beskriver en variabler, mens metoder beskriver funktioner. Eksempler på definitionen af en klasse fremgår af \autoref{tab:klasse}. 

\begin{table}[H]
\centering
\begin{tabular}{|l|l|}
\hline
\textbf{Egenskaber} & \textbf{Metoder} \\ \hline
\begin{tabular}[c]{@{}l@{}}Navn \\ Køn\\ Alder\\ Højde \\ Vægt\end{tabular} & \begin{tabular}[c]{@{}l@{}}Gå\\ Løbe\\ Hoppe\\ Sove\\ Tale\end{tabular} \\ \hline
\end{tabular}
\caption{Definition af en klasse. Klassen består af egenskaber og metoder.}
\label{tab:klasse}
\end{table}

\noindent
OOP består af tre grundprincipper indkapsling, nedarvning og polymorfi. 