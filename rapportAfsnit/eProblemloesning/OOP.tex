I dette kapitel beskrives de metoder, der anvendes i problemløsningen, herunder de grundlæggende principper inden for objektorienteret programmering samt forskellige diagrammer, der anvendes inden for dette. Derudover beskrives modeller, der kan anvendes til udviklingen af app's.

\section{Objektorienteret programmering} \label{sec:OOP}
Objektorienteret programmering er et programmeringsparadigme, som anvendes til at analysere, designe, implementere samt udvikle app's. Hyppige termer inden for objektorienteret programmering er blandt andet objekter, klasser, indkapsling, nedarvning og polymorfi.\cite{Stefanov2013,Brahma2015}

I objektorienteret programmering opdeles programmeringskoden i klasser, hvor hver klasse fungerer som en opskrift for et objekt. Hvert objekt er en instans af en bestemt klasse, hvor en klasse kan være bygget op omkring en eller flere instanser. De forskellige objekter repræsenterer hver sin del af app'en og indeholder data og logik. Derudover har objekterne mulighed for at kommunikere mellem hinanden. Objekter er karakteriseret ud fra deres egenskaber, og deres funktioner er beskrevet ved metoder.\cite{Stefanov2013,Brahma2015} Eksempler på egenskaber og metoder fremgår af \autoref{tab:objekt}. 

\begin{table}[H]
\centering
\begin{tabular}{|l|l|}
\hline
\textbf{Egenskaber} & \textbf{Metoder} \\ \hline
\begin{tabular}[c]{@{}l@{}}Navn \\ Køn\\ Alder\\ Højde \\ Vægt\end{tabular} & \begin{tabular}[c]{@{}l@{}}Gå\\ Løbe\\ Hoppe\\ Sove\\ Tale\end{tabular} \\ \hline
\end{tabular}
\caption{Objekter karakteriseres ud fra deres egenskaber som for eksempel navn, mens metoder beskriver deres funktion som for eksempel sove.}
\label{tab:objekt}
\end{table}

\noindent
Objektorienteret programmering består af tre grundprincipper, herunder indkapsling, nedarvning og polymorfi. Indkapsling er en illustration af, at objekter både indeholder egenskaber og metoder. Egenskaber opbevarer data, mens metoder anvendes til at behandle data. Indkapsling kan både have synlige og skjulte informationer. Synlig information udgør ofte grænsefladen, såsom knapper og display, mens skjult information kan være implementeringen af grænsefladen. Dette gør sig også gældende for objekter, hvilket defineres som public eller private. Ved public har alle objekter adgang til metoderne, mens private kun er metoder med samme objekt, der kan tilgå denne. Nedarvning betyder, at et objekt kan arve data og funktioner fra et andet objekt. Dette muliggør, at objektet kan udvides med ekstra data og funktioner. Polymorfi giver mulighed for, at to klasser kan have samme grænseflade. Denne er defineret ved nedarvningen.\cite{Stefanov2013}
