\chapter{Metode}
\section{Objektorienteret programmering}
Objektorienteret programmering(OOP) er et programmeringsparadigme, som anvendes til at analysere, designe, implementere samt udvikle app's. De mest anvendte termer inden for OOP er objekter, klasser, indkapsling, nedarvning og polymorfi. 

I OOP opdeles programmeringskoden i klasser, hvor hver klasse fungerer som en opskrift for et objekt. Hvert objekt er en instans af en bestemt klasse, hvor en klasse kan være bygget op omkring en eller flere instanser. De forskellige objekter repræsenterer hver sin del af app'en og indeholder hver sin data og logik. Derudover har objekterne mulighed for at kommunikere mellem hinanden. Objekter er karakteriseret ud fra dens egenskaber og funktionen af objektet er beskrevet ved metoder. Eksempler på egenskaber og metoder fremgår af \autoref{tab:objekt}. 

\begin{table}[H]
\centering
\begin{tabular}{|l|l|}
\hline
\textbf{Egenskaber} & \textbf{Metoder} \\ \hline
\begin{tabular}[c]{@{}l@{}}Navn \\ Køn\\ Alder\\ Højde \\ Vægt\end{tabular} & \begin{tabular}[c]{@{}l@{}}Gå\\ Løbe\\ Hoppe\\ Sove\\ Tale\end{tabular} \\ \hline
\end{tabular}
\caption{Objekter karakteriseres ud fra dens egenskaber som f.eks. navn, mens metoder beskriver funktionen som f.eks. sove.}
\label{tab:objekt}
\end{table}

\noindent
OOP består af tre grundprincipper, herunder indkapsling, nedarvning og polymorfi. Indkapsling er en illustration af, at objekter både indeholder egenskaber og metoder. Egenskaberne opbevarer data, mens der anvendes metoder til at behandle data. Indkapsling kan både have synlige og skjulte informationer. Synlig information udgør  ofte grænsefladen, såsom knapper og display, mens skjult information kan være implementeringen af grænsefladen. Dette gør sig også gældende for objekter, hvilket defineres som public eller private. Ved public har alle objekter  adgang til denne, mens private kun er metoder med samme objekt der kan tilgå denne. Nedarvning betyder, at et objekt kan arve data og funktioner fra et andet objekt. Dette muliggør, at objektet kan udvides med ekstra data og funktioner. Polymorfi giver mulighed for at to klasser kan have samme grænseflade. Denne er defineret ved nedarvningen.  
