\chapter{Lungernes anatomi og fysiologi}
\section{Lungernes opbygning}
Kroppens to lunger ligger i thoraxhulen og er omgivet af thorax. Thorax består sternum og af de ydre og indre interkostal muskler herunder thorakal hvirvlerne, ribbenene samt musklerne mellem ribbenene. Halsmuskler udgør loppen af sternum og diafragma udgør bunden. De forskellige muskler og diafragma har flere funktioner under ventilation. 

\subsection{Ventilation}
Ventilation beskriver transport af luft frem og tilbage mellem atmosfæren og lungealveolerne. Luften bevæger sig fra områder med højt tryk til områder med lavt tryk. Det atmosfæriske tryk ændres som regel ikke, hvilket medfører, at det er variationer i trykket i alveolerne, der sørger for transporten af luft frem og tilbage. Trykket skabes ved at lungerne udvides og presses sammen, hvorved alveoletrykket bliver lavere eller højere end det atmosfæriske tryk alt efter om der er tale om inspiration og eksspiration. 

\subsubsection{Inspiration}

\subsubsection{Eksspiration}