\chapter{Anatomi og fysiologi}
\section{Lungernes anatomi og fysiologi}
Kroppens to lunger ligger i thoraxhulen og er omgivet af thorax. Thorax består af sternum og de ydre og indre interkostal muskler, herunder thorakal hvirvlerne, ribbenene samt musklerne mellem ribbenene. Halsmuskler udgør loftet og diafragma udgør gulvet af sternum. De forskellige muskler og diafragma benyttes ved ventilation.

\subsection{Ventilation}
Ventilation beskriver transport af luft frem og tilbage mellem atmosfæren og lungealveolerne. Luften bevæger sig fra områder med højt tryk til områder med lavt tryk. Det atmosfæriske tryk ændres ikke, hvilket medfører, at det er variationer i trykket i alveolerne, der sørger for transporten af luft frem og tilbage. Trykket skabes ved at lungerne udvides og presses sammen, hvorved alveoletrykket bliver lavere eller højere end det atmosfæriske tryk alt efter om der er tale om inspiration eller eksspiration. 

\subsubsection{Inspiration}
Inden inspiration er alle interkostal musklerne afslappede og alveoletrykket er det samme som det atmosfæriske tryk, hvilket betyder at der ikke strømmer luft gennem luftvejene. Ved inspiration udvides thoraxhulen, som medvirker til at trykket inde i lungen falder, hvormed lungerne udvider sig. Kontraktion af interkostal musklerne øger volumen i både bredde og dybde, og diafragma, der trækker sig sammen, medfører at thoraxhulens volumen øges. Halsmusklerne hæver ribbenene, hvilket yderligere medvirker til øget volumen af thoraxhulen. Når inspirationen er afsluttet, afslappes interkostal musklerne igen. 

\subsubsection{Eksspiration}
Eksspiration skyldes at de elastiske kræfter i lungerne og sternum presser lungerne sammen, når interkostal musklerne afslappes. Alveoletrykket inde i lungerne er større end det atmosfæriske tryk uden for. For at udligne dette presses diafragma op mod sternum og interkostal musklerne sig sammen og trækker ribbene nedad, hvormed lungernes volumen mindskes. Eksspirationen fortsættes indtil trykforskellen mellem det atmosfæriske tryk og alveoletrykket er udlignet.  

\subsection{Respirationssystem}
Respirationssystemet består af et øvre og et nedre. Det øvre respirationssystem består af næsen, næsehulen, paranasale sinuser (paranasal sinuses) og svælget. Det nedre respirationssystem indeholder larynx, trachea, bronkier, bronkioler og lungealveoler. 

\subsubsection{Transport af ilt}
Ilten indtages i det øvre respirationssystem gennem næsen eller munden, herefter transporteres luften gennem trachea, hvor den deler sig i to hovedebronkier. Hovedebronkierne deler sig igen i mindre og mindre bronkier, hvor de til sidst ender i alveoler. Disse har udposninger yderst, kaldet bronkioler, og er omgivet af lungekapillærer. Alveolevæggen er tynd, hvilket medfører til at ilten kan trænge ind og over i blodet gennem lungekapillærer, hvorved blodet iltes og kan transporteres ud i musklerne. Modsat kan affaldsstoffet, kuldioxid, som frigives ved eksspiration, trænge fra blodet over i alveolerne, hvorved det udskilles ved eksspiration. 


\subsection{Overgang}
.... Mangler en overgang ... ved heller ikke om dette skal stå her .. kan være det skal ind i selve problemanalysen . nu er det skrevet


\subsubsection{Kronisk bronkitis}
Kronisk bronkitis skyldes at bronkierne i slimhinden er beskadiget. Der produceres for meget slim og der er for få fimrehår, også kaldet cillia, til at transportere slim og støvpartikler op i svælget. Ved kronisk bronkitis er slimhinden også fortrykket, hvilket sammen med den øgede slim medfører mindre plads til luft i bronkierne og problemer med at hoste slim op, hvilket kan medvirke til forsnævring af små og store bronkier. 

\subsubsection{Emfysem}
Emfysem skyldes at lungernes volumen er øget grundet beskadiget lungevæv herunder destruktion af elastiske fibre og nedbrydning af væggene i de små lungeblærer. Dette medfører til at overfladen, som lungen har til rådighed ved luftudvekslingen med blodet, mindskes og små bronkier kan klappe sammen og derved lukke under ventilation.  
