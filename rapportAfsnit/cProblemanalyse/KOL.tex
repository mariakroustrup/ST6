\section{Kronisk obstruktiv lungesygdom}
Kronisk obstruktiv lungesygdom (KOL) er en kronisk inflammatorisk sygdom, der resulterer i gradvist nedsat lungefunktion. Den kroniske inflammation opstår i luftvejene og lungevævet, hvilket forårsager, at bronkiernes vægge ødelægges og/eller luftvejene forsnævres.\cite{Basisbogen2016} KOL er beslægtet med to patologier, herunder kronisk bronkitis og emfysem. KOL-patienter oplever ofte begge patologier, men omfanget af disse varierer fra patient til patient.\cite{Basisbogen2016,Healthguidances2016}

Kronisk bronkitis er luftvejsinflammation, hvor bronkierne i slimhinden er beskadiget, hvilket medfører en øget slimproduktion. Derudover er antallet af cillia mindsket, hvormed transport af slim og støvpartikler fra bronkierne til svælget begrænses, hvorfor der opstår bakterielle infektioner. \cite{Frausing2011, Britannica2016}. Patienter med overvejende kronisk bronkitis betegnes blue bloater. Disse patienter har ofte lungeinfektioner, cor pulmonale og har type 2 respirationssvigt. Cor pulmonale betegner en trykbelastet og med tiden udvidet hypertrofisk samt dårlig fungerende højre ventrikel. Type 2 respirationssvigt betegner et lavt iltniveau og højt indhold af kuldioxid. Den dårlige ilttilførsel til ekskrementer, huden samt læber vil medvirke til, at huden bliver blålig, hvorfor disse patienter omtales blue bloater. \cite{Healthguidances2016}

Emfysem karakteriseres ved lunge destruktion, dannelse af lungeblærer, tab af elastisk tilbagetrækning og hyperinflation. Emfysem skyldes, at lungernes volumen er øget grundet beskadiget lungevæv, herunder destruktion af elastiske fibre og nedbrydning af væggene i de små lungeblærer. Dette medfører til, at overfladen, som lungerne har til rådighed ved luftudvekslingen mindskes, hvorved små bronkier kan klappe sammen og derved lukke under ventilation.\cite{Frausing2011a,Flaschen-Hansen2008} Patienter med overvejende emfysem betegnes pink puffer. Disse patienter har ofte kakeksi, tøndeformet brystkasse og har type 1 respirationssvigt. Kakeksi er betegnelsen for alvorlig afmagring eller vægttab med tydelige tegn på nedbrydning af muskelmasse og fedtvæv. Type 1 respirationssvigt opnås ved lavt iltniveau og normal indhold af kuldioxid. Ved vejrtrækning vil patienters krop blive pustet op og huden vil blive rødlig, hvorfor disse patienter omtales pink puffer.\cite{Healthguidances2016}

Definitionen af KOL beskrives ved ratioen mellem forceret eksspiratorisk volumen i et sekund (FEV1) og forceret vitalkapacitet (FVC). FEV1 måles ud fra, hvad der udåndes i det første sekund efter en maksimal indånding. FVC er en indikator for lungevolumen målt i liter. Ved tilfælde af KOL er FEV1/FVC under 70 \% af den forventede lungekapacitet. \cite{Basisbogen2016}


Der er flere disponerende faktorer til KOL heriblandt skadelige partikler samt gasser, miljøpåvirkninger og genetiske faktorer. Den hyppigste årsag til KOL er rygning, som fremskynder tab af lungefunktionen.\cite{dsam2016,Basisbogen2016,Martinez2016} Foruden rygning kan miljøpåvirkninger have betydning for udviklingen af KOL. Opvækst i et dårligt miljø vil kunne påvirke barnets lunger til ikke at udvikle sig ordentligt, hvilket kan resultere i en lavere FEV1. Dertil vil et dårligt arbejdsmiljø kunne medvirke til en accelererende reduktion i FEV1, der ligeledes kan øge risikoen for KOL. \cite{Martinez2016} Dette betyder, at en lav eller en accelererende reduktion af FEV1 vil mindske FEV1/FVC-ratioen.  
