I dette kapitel beskrives kronisk obstruktiv lungesygdom og de tilhørende symptomer. Yderligere undersøges det, hvordan KOL diagnosticeres samt, hvilke behandlingsmuligheder KOL-patienter tilbydes. Heraf analyseres KOL-patienters resultater efter gennemgået rehabiliteringsforløb.

\section{Kronisk obstruktiv lungesygdom}
Kronisk obstruktiv lungesygdom (KOL) er en kronisk inflammatorisk sygdom, der resulterer i gradvist nedsat lungefunktion. Inflammationen opstår i luftvejene og lungevævet, hvilket forårsager, at bronkiernes vægge ødelægges og/eller luftvejene forsnævres. På nuværende tidspunkt er KOL den tredje hyppigste dødsårsag på verdensplan \cite{WHO2017}. I Danmark er der ca. 430.000 patienter diagnosticeret med KOL, hvortil der er en årlig mortalitet på 3.500 patienter, hvilket gør KOL til den fjerde hyppigste dødsårsag i Danmark \cite{Basisbogen2016}.


KOL er beslægtet med to patologier, herunder kronisk bronkitis og emfysem. KOL-patienter oplever ofte begge patologier, men omfanget af disse varierer fra patient til patient.\cite{Basisbogen2016}
Kronisk bronkitis er luftvejsinflammation, hvor bronkierne i slimhinden er beskadiget, hvilket medfører en øget slimproduktion. Derudover er antallet af cilia mindsket, hvormed transport af slim og støvpartikler fra bronkierne til svælget begrænses, hvorfor der opstår bakterielle infektioner.\cite{Frausing2011,Britannica2016} KOL-patienter med overvejende kronisk bronkitis betegnes blue bloater. Disse patienter har ofte lungeinfektioner, cor pulmonale, hvilket betegner en trykbelastet og med tiden udvidet hypertrofisk samt dårlig fungerende højre ventrikel. Derudover oplever patienter ofte  type 2 respirationssvigt, hvor iltniveauet er lavt og indhold af kuldioxid højt. Den dårlige ilttilførsel til ekstremiteter, huden samt læber vil medvirke til, at huden bliver blålig, hvorfor disse patienter omtales blue bloater.\cite{Healthguidances2016}

Emfysem skyldes, at lungernes volumen er øget grundet beskadiget lungevæv, herunder destruktion af elastiske fibre og nedbrydning af væggene i de små lungeblærer. Dette medfører, at overfladen som lungerne har til rådighed ved luftudvekslingen mindskes, hvormed små bronkier kan klappe sammen og derved lukke under ventilation.\cite{Frausing2011a,Flaschen-Hansen2008} KOL-patienter med overvejende emfysem betegnes pink puffer. Disse patienter lider ofte af alvorlig afmagring eller vægttab med tydelige tegn på nedbrydning af muskelmasse og fedtvæv. Deres brystkasse er tøndeformet og de oplever type 1 respirationssvigt. Type 1 respirationssvigt betegner et lavt iltniveau og normalt indhold af kuldioxid. Disse patienter omtales pink puffer, da deres kroppe ved vejrtrækning pustes op og huden bliver rødlig.\cite{Healthguidances2016}

KOL bestemmes ved ratioen mellem forceret eksspiratorisk volumen (FEV1) og forceret vitalkapacitet (FVC). FEV1 måles ud fra, hvad der udåndes i det første sekund efter en maksimal indånding. FVC er lungevolumen målt i liter. Ved tilfælde af KOL er FEV1/FVC under 70 \% af den forventede lungekapacitet.\cite{Basisbogen2016}

Der er flere disponerende faktorer til KOL heriblandt skadelige partikler samt gasser, miljøpåvirkninger og genetiske faktorer. Den hyppigste årsag til KOL er tobaksrygning, som fremskynder tab af lungefunktionen.\cite{dsam2016,Basisbogen2016,Martinez2016} Foruden tobaksrygning kan miljøpåvirkninger have betydning for udviklingen af KOL. Opvækst i et dårligt miljø vil kunne påvirke barnets lunger til ikke at udvikle sig ordentligt, hvilket kan resultere i en lavere FEV1. Derudover vil et dårligt arbejdsmiljø, som f.eks. arbejde med asbest, kunne medvirke til en accelererende reduktion i FEV1, der ligeledes kan øge risikoen for KOL.\cite{Martinez2016} 
%Dette betyder, at en lav eller en accelererende reduktion af FEV1 vil mindske FEV1/FVC-ratioen.  
