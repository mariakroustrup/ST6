\section{Applikationer med relation til KOL-patienter}
På nuværende tidspunkt findes forskellige danske applikationer (app), herunder KOL-guidelines, Kronika og HomeRehab, der kan anvendes af  sundhedspersonale og patienter som et hjælpemiddel til forståelsen samt rehabiliteringen af KOL.\cite{KOLguidelines2012,Kronika2014,HealthcareDenmark2017} 

KOL-guidelines er udviklet af Boehringer Ingelheim, der er baseret på guidelines fra Dansk Lungemedicinsk Selskab (DLS), og anvendes som led i at klassificere sværhedsgraden af KOL ud fra antal af eksacerbationer, MRC og CAT. Ud fra sværhedsgraden kommer appen med anbefalede behandlingsmuligheder med henblik på fysisk træning, farmakologisk behandling, vaccination og essentielle elementer i symptomlindring.\cite{KOLguidelines2012}    

Kronika, som er udviklet af Region Hovedstaden giver generel viden omkring forskellige kroniske sygdomme heriblandt hjertekrampe, hjertesvigt, KOL, tryksår og type 2 diabetes. I relation til KOL giver appen viden omkring sygdommen og vejledninger om vejrtrækningsteknikker samt stillinger, der reducerer åndenød. Ligeledes fremhæves generelle anbefalinger som fysisk aktivitet samt,  hvilke sygdomsmarkører, der fremkommer ved optakt til en eventuel sygdomsforværrelse.\cite{Kronika2014}    

De ovennævnte apps vil primært give sundhedspersonalet et overblik over, hvilke anbefalinger, der gives til KOL-patienter. Disse fungerer dermed som opslagsværk for sygdommen. I relation til opretholdelse af resultaterne fra rehabiliteringen, anses de ikke til at have en gavnlig effekt. 
Dette skyldes blandt andet, at de nævnte apps ikke er udviklet til patienter, men for det varetagende sundhedspersonale.\cite{Kronika2014} \fxnote{Ud fra tidligere nævnte faktorer for vedligeholdelse af rehabiliteringsresultater i afsnit XX, ville en app blandt andet skulle være i stand til at motivere patienten til at være fysisk aktiv i hverdagen, hvilket ikke fremkommer hos de førnævnte apps.} 

HomeRehab, udviklet af Firmaet Aidcube, kan anvendes under og efter rehabiliteringsforløbet. HomeRehab har til formål at gøre KOL-patienter i stand til at tage varetage sig selv ved at opretholde effekterne af rehabiliteringen gennem motivering til daglig træning. Data fra HomeRehab appen hjælper også sundhedspersonale, der kan tilgå data via en webportal, med at identificere tegn på sygdomsforværring, hvilket anvendes til at reducere risikoen for hospitalsindlæggelse. HomeRehab testes på nuværende tidspunkt i samarbejde med blandt andet Hvidovre Hospital, Frederiksberg Hospital og Silkeborg Kommune. 
I testperioden bliver en behandlingsgruppe introduceret i anvendelse af HomeRehab i et 7 ugers rehabiliteringsforløb, hvor de efterfølgende anvender appen hjemme de efterfølgende 6 måneder.\cite{HealthcareDenmark2017}

På nuværende tidspunkt er der ikke udgivet forskningsartikler omhandlende effekterne fra anvendelsen af HomeRehab appen, dog fremgår resultater fra prøveperioden i Silkeborg Kommune på Aidcubes hjemmeside. Dertil oplyses, at 68 \% af KOL-patienterne i behandlingsgruppen opretholder deres daglige aktivitet og har medført 10 \% reduktion i KOL-relaterede hospitalsindlæggelser.\cite{AidCube2017}
