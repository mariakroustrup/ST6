\section{Rehabilitering}
Pulmonary rehabilitation er en måde at forsøge at reducere symptomer forbundet med KOL. 
KOL-patienter kan henvises til et rehabiliteringsforløb gennem egen læge eller fra et hospital [abotekrehabilitering].Varigheden af rehabiliteringen er anbefalet til at forløbe over en 8 ugers periode, for således at rehabiliteringen har nogen gavnlig effekt \cite{McCarthy2015}. 
Rehabiliteringen kan give patienten bedre mulighed for deltagelse i hverdagen. Såfremt at patientens tilstand tillader det, anbefales rehabilitering til de fleste KOL-patienter, og omhandler typisk tobaksafvænning, fysisk træning, patientuddannelse, træning af dagligdagsaktiviteter og ernæringsvejledning. \cite{McCarthy2015,Habraken2011} [sundkol2015,] 

Individuel rehabilitering anses som værende fundamental for KOL-patienter, hvor forløbet tilpasses ift. patientens behov med henblik at opnå det bedste udbytte af rehabiliteringen \cite{McCarthy2015,Habraken2011} [sundkol2015]. Ligeledes vurderes rehabiliteringen på baggrund af graden af KOL, da KOL er en heterogen sygdom. Dette betegnes ud fra, at KOL fremkommer i flere grader/variationer samt med varierende progression \cite{McCarthy2015}. 

Tobaksafvænning er, som beskrevet i afsnit XX (Linettes), et relevant element ift. at begrænse udviklingen af sygdommen, og bevare mest mulig lungefunktion. Gennem motionen, der udføres under rehabiliteringen, bliver patienten bedre til at benytte den resterende lungefunktion, samt opnå et bedre fysisk funktionsniveau [sundkol2015]. 
Træningen ville ligeledes modvirke eventuelle følger ved KOL, da motion øger muskelfunktionen samt udsætter træthed, hvilket medfører forøget aktivitetstolerance \cite{McCarthy2015}. Dog kan motion resultere i åndenød hos nogen KOL-patienten, der kan forstærkes, hvis patienten påvirkes af angst som følge af åndenøden. For KOL-patienten kan frygten for angsten, betyde vedkommende afholder sig fra fysisk aktivitet, der endvidere kan give en progredierende svækkelse i tolerancen af fysisk aktivitet \cite{McCarthy2015} [sundKOL2015]. Den fysiske svækkelse kan endvidere resultere i, at patienten bliver socialt isoleret. [sundkol2015]
  
Et led i rehabiliteringen er ligeledes, at patienten opnår viden indenfor sygdomshåndtering, der omhandler kendskab og forebyggelse til sygdommen, livsstilsændringer og håndtering af eksacerbationer\cite{McCarthy2015} [sundKOL2015]. Her fokuseres bl.a. på de gavnlige effekter ved rygestop og regelmæssig motion, samt hvornår og hvordan evt. medicin skal indtages. Patienten vil yderligere blive introduceret til energibesparende strategier og vejrtrækningsøvelser \cite{McCarthy2015} [sundkol2015].   

I Danmark kan patienter henvises fra sygehuslæge eller egen læge til et rehabiliteringsforløb i et sundhedscenter eller på et hospital. Rehabiliteringsforløbene varer typisk 7-8 uger, hvor KOL-patienter møder til træning 1-2 gange om ugen. De resterende dage vil patienten træne hjemmefra. [lunge.dk/rehabilitering] 
Såfremt at frekvensen og intensiteten er den samme, vil effekten være tilsvarende uanset om træningen foretages ude eller hjemme [GOLD]. Opfølgninger kan foretages efter afslutning af rehabiliteringsforløbet, for at undersøge om patienten efterfølgende opretholder de gavnlige effekter [lunge.dk/rehabilitering]. 