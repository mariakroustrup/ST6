\section{Rehabilitering}
Da KOL er kronisk lungesygdom, hvorved den tabte lungefunktion ikke kan genoprettes, tilbydes KOL-patienter rehabilitering med henblik på at mindske deres symptomer. 

I Danmark henvises KOL-patienter til rehabilitering fra et hospital eller praktiserende læge. Rehabilitering forløber typisk over en periode på $7$-$8$ uger på et sundhedscenter eller på et hospital. KOL-patienter møder til træning $1$-$2$ gange om ugen, de resterende dage vil patienten kunne træne de fremviste øvelser hjemmefra. \cite{McCarthy2015}[lunge.dk/rehabilitering] 

Individuel rehabilitering anses som værende fundamental for KOL-patienter, hvor forløbet tilpasses i forhold til patienternes behov med henblik at opnå det bedste udbytte af rehabiliteringen \cite{McCarthy2015,Habraken2011}[sundkol2015]. Ligeledes vurderes rehabiliteringen på baggrund af graden af KOL, da KOL er en heterogen sygdom. Dette betegnes ud fra, at KOL fremkommer i flere grader samt med varierende progression \cite{McCarthy2015}. 

Rehabiliteringen kan give patienterne bedre mulighed for deltagelse i hverdagen, såfremt patienternes tilstand tillader det. \cite{McCarthy2015,Habraken2011} [sundkol2015] Opfølgninger kan foretages efter afslutning af rehabiliteringsforløbet, for at undersøge om patienten efterfølgende opretholder de gavnlige effekter [lunge.dk/rehabilitering]. s


\subsection{Rehabiliteringsforløbet}
Rehabiliteringsforløbet fokuserer på tobaksafvænning, fysisk træning, træning af dagligdags aktiviteter, kendskab til sygdommen samt ernæringsvejledning. \cite{McCarthy2015,Habraken2011} [sundkol2015] 

Tobaksafvænning er, som beskrevet i \autoref{sec:behandling}, et relevant element i forhold til at begrænse udviklingen af sygdommen og bevare mest mulig lungefunktion. Den fysiske træning, der udføres under rehabiliteringen, medvirker til, at patienterne kan opnå et bedre udbytte af den resterende lungefunktion, samt opnå et bedre fysisk funktionsniveau [sundkol2015]. 
Træningen vil ligeledes modvirke eventuelle følger ved KOL, da fysisk træning øger muskelfunktionen samt udsætter træthed, hvilket medfører forøget aktivitetstolerance \cite{McCarthy2015}. Dog kan den fysiske træning resultere i åndenød hos KOL-patienter, der kan forstærkes, hvis patienten påvirkes af angst som følge af åndenøden. Dette kan betyde, at KOL-patienter afholder sig fra fysisk træning på grund af frygten for angst. \cite{McCarthy2015} [sundKOL2015]. 

Et led i rehabiliteringen er ligeledes, at patienten opnår viden indenfor sygdomshåndtering, der omhandler kendskab til og forebyggelse af sygdommen, livsstilsændringer og håndtering af eksacerbationer\cite{McCarthy2015} [sundKOL2015]. Her fokuseres blandt andet på de gavnlige effekter ved rygestop og regelmæssig fysisk aktivitet, samt hvornår og hvordan eventuel medicin skal indtages. Patienten vil yderligere blive introduceret til energibesparende strategier og vejrtrækningsøvelser \cite{McCarthy2015} [sundkol2015].   

