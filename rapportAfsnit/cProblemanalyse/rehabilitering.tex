\section{Rehabilitering af KOL-patienter}
Da KOL er en kronisk lungesygdom tilbydes KOL-patienter rehabilitering med henblik på at mindske deres symptomer. 

I Danmark henvises KOL-patienter til rehabilitering fra praktiserende læge eller hospital, hvor rehabiliteringen typisk forløber over en otte ugers periode på et sundhedscenter eller hospital. Under dette forløb tilbydes KOL-patienter træning en til to gange om ugen, de resterende dage vil patienter kunne udføre fremviste øvelser hjemme.\cite{McCarthy2015,Frausing2011b} Som tidligere nævnt kan den tabte lungefunktion ikke genoprettes, dog kan motion nedsætte symptomerne som følge af KOL. Træning styrker patienters muskler samt forbedrer deres kondition, herved vil vejrtrækningen forbedres, da lungerne fremover belastes mindre ved fysisk aktivitet.\cite{Lungeforeningen2016}

Individuel rehabilitering anses som værende fundamental for KOL-patienter, hvor forløbet tilpasses patienters behov med henblik på at opnå det bedste udbytte af rehabiliteringen.\cite{McCarthy2015,Habraken2011,Sundhedsstyrelsen2015} Ligeledes vurderes rehabiliteringen på baggrund af graden af KOL, da KOL fremkommer i flere grader samt med varierende progression \cite{McCarthy2015}.

Rehabiliteringen kan give patienter bedre mulighed for deltagelse i hverdagen, såfremt patienters tilstand tillader det \cite{McCarthy2015,Habraken2011, Sundhedsstyrelsen2015}. Opfølgninger kan foretages efter rehabiliteringsforløbet er afsluttet, for således at undersøge om patienter opretholder de gavnlige effekter \cite{Frausing2011b}.


\subsection{Rehabiliteringsforløb}
Rehabiliteringsforløbet fokuserer på tobaksafvænning, fysisk træning, kendskab til sygdommen samt ernæringsvejledning \cite{McCarthy2015,Habraken2011,Sundhedsstyrelsen2015}.

Tobaksafvænning er, som beskrevet i \autoref{sec:behandling}, et relevant element i forhold til at begrænse udviklingen af sygdommen og bevare mest mulig lungefunktion. Den fysiske træning, der udføres under rehabiliteringen, medvirker til, at patienter kan opnå et bedre udbytte af den resterende lungefunktion samt opnå et bedre fysisk funktionsniveau.\cite{Sundhedsstyrelsen2015}
Træningen kan ligeledes modvirke eventuelle følger ved KOL, da fysisk træning øger muskelfunktionen samt udsætter træthed, hvilket medfører øget aktivitetstolerance \cite{McCarthy2015}. Den fysiske træning kan dog resultere i åndenød hos KOL-patienter, der kan forstærkes, hvis patienter påvirkes af angst som følge af åndenød. Dette kan betyde, at KOL-patienter afholder sig fra fysisk træning på grund af frygten for angst.\cite{McCarthy2015, Sundhedsstyrelsen2015} 

Et led i rehabiliteringen er ligeledes, at patienter opnår viden indenfor sygdomshåndtering, der omhandler kendskab til og forebyggelse af sygdommen, livsstilsændringer samt håndtering af eksacerbationer. Her fokuseres blandt andet på de gavnlige effekter ved tobaksophør og regelmæssig fysisk aktivitet, samt hvornår og hvordan eventuel medicin skal indtages. Patienten vil yderligere blive introduceret til energibesparende strategier og vejrtrækningsøvelser.\cite{McCarthy2015,Sundhedsstyrelsen2015}    

\section{Efter rehabiliteringsforløb}
Gennem studier er det oplyst, at ikke alle patienter er i stand til at opretholde resultaterne, og deres fysiske tilstand falder tilbage til niveauet før rehabiliteringsforløbet \cite{Egan2012,Beachamp2013,Zanaboni2017,Ringbaek2008}. 
Årsagerne til dette tilbagefald kan blandt andet være som følge af, at rehabiliteringen ikke er med til at gøre patienter mere aktive i hjemmet efter afsluttet forløb, da de falder tilbage til deres tidligere vaner og rutiner \cite{Egan2012}. Ligeledes ses det hos patienter, der fortsat træner, at intensiteten og hyppigheden af træningen falder \cite{Ringbaek2008}. 
Dansk Selskab for Almen Medicin (DSAM) anbefaler dertil KOL-patienter
at følge et vedligeholdelsesprogram bestående af fire til fem træningssessioner om ugen
efter afsluttet rehabiliteringsforløb \cite{dsam2016}. Derudover tilbydes KOL-patienter at deltage i forskellige træningssessioner og fællesskaber, hvor de har mulighed for at danne træningsgrupper og afholde arrangementer \cite{Sundhedsstyrelsen2015}. Hertil har Lungeforeningen i Danmark forskellige lokalafdelinger, hvor der et par gange årligt afholdes arrangementer for patienter samt pårørende \cite{Lungeforeningen2016}.
Fordele ved gruppeaktiviteter er, at KOL-patienter kan undgå social isolation samtidig med, at de lærer af hinandens erfaringer i forhold til, hvordan de hver især oplever og håndterer sygdommen. Herved kan sociale fællesskaber være en medhjælpende faktor til vedligeholdelse af effekten ved rehabiliteringen.\cite{dsam2016}

Det ses i stigende grad, at applikationer (apps) anvendes i sundhedsrelateret sammenhæng for at skabe en forbindelse mellem professionel behandling og self-management \cite{Williams2014}. Dertil ses app'en HomeRehab, der har til formål at gøre KOL-patienter i stand til at tage varetage sig selv ved at opretholde effekterne af rehabiliteringen gennem motivering til daglig træning. Denne app er udviklet af Firmaet Aidcube til anvendelse under og efter et rehabiliteringsforløb. Data fra HomeRehab app'en hjælper også sundhedspersonale, der kan tilgå data via en webportal, med at identificere tegn på sygdomsforværring, hvilket anvendes til at reducere risikoen for hospitalsindlæggelse. HomeRehab testes på nuværende tidspunkt i samarbejde med blandt andet Hvidovre Hospital, Frederiksberg Hospital og Silkeborg Kommune.\cite{HealthcareDenmark2017} 

\section{Projektafgrænsning}
I dette projekt fokuseres der på KOL-patienter samt deres formåen til at reducere deres symptomer. KOL-patienter tilbydes rehabiliteringsforløb for at få viden om sygdommen, hjælp til tobaksophør samt ernæring og motion. Rehabiliteringsforløb har til formål at nedsætte symptomerne, således en bedre livskvalitet kan opnås. Studier viser dog, at KOL-patienter har svært ved at opretholde resultaterne efter et afsluttet rehabiliteringsforløb. I Danmark ses forskellige værktøjer til at forsøge at opretholde resultaterne, herunder ses blandt andet vedligeholdelsesprogrammer, sociale fællesskaber samt forskellige app's. De sociale fælleskaber viser sig positivt i forhold til motivation for opretholdelse af den forbedrede livsstil. 
På baggrund af dette udvikles en app med fokus på social interaktion og motivation til vedligeholdelse af resultaterne fra rehabiliteringsforløb. 
