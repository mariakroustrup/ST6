\section{Rehabilitering af KOL-patienter}
Da KOL er kronisk lungesygdom, hvorved den tabte lungefunktion ikke kan genoprettes, tilbydes KOL-patienter rehabilitering med henblik på at mindske deres symptomer. 

I Danmark henvises KOL-patienter til rehabilitering fra praktiserende læge eller hospital, hvor rehabiliteringen typisk forløber over otte ugers periode på et sundhedscenter eller hospital. Under dette forløb tilbydes KOL-patienter træning en til to gange om ugen, de resterende dage vil patienter kunne udføre fremviste øvelser hjemme.\cite{McCarthy2015,Frausing2011b} 

Individuel rehabilitering anses som værende fundamental for KOL-patienter, hvor forløbet tilpasses patienters behov med henblik på at opnå det bedste udbytte af rehabiliteringen.\cite{McCarthy2015,Habraken2011,Sundhedsstyrelsen2015} Ligeledes vurderes rehabiliteringen på baggrund af graden af KOL, da KOL fremkommer i flere grader samt med varierende progression \cite{McCarthy2015}.

Rehabiliteringen kan give patienter bedre mulighed for deltagelse i hverdagen, såfremt patienters tilstand tillader det \cite{McCarthy2015,Habraken2011, Sundhedsstyrelsen2015}. Opfølgninger kan foretages efter rehabiliteringsforløbet er afsluttet, for således at undersøge om patienter opretholder de gavnlige effekter \cite{Frausing2011b}.


\subsection{Rehabiliteringsforløb}
Rehabiliteringsforløbet fokuserer på tobaksafvænning, fysisk træning, kendskab til sygdommen samt ernæringsvejledning \cite{McCarthy2015,Habraken2011,Sundhedsstyrelsen2015}.

Tobaksafvænning er, som beskrevet i \autoref{sec:behandling}, et relevant element i forhold til at begrænse udviklingen af sygdommen og bevare mest mulig lungefunktion. Den fysiske træning, der udføres under rehabiliteringen, medvirker til, at patienter kan opnå et bedre udbytte af den resterende lungefunktion samt opnå et bedre fysisk funktionsniveau.\cite{Sundhedsstyrelsen2015}
Træningen kan ligeledes modvirke eventuelle følger ved KOL, da fysisk træning øger muskelfunktionen samt udsætter træthed, hvilket medfører øget aktivitetstolerance \cite{McCarthy2015}. Den fysiske træning kan dog resultere i åndenød hos KOL-patienter, der kan forstærkes, hvis patienter påvirkes af angst som følge af åndenød. Dette kan betyde, at KOL-patienter afholder sig fra fysisk træning på grund af frygten for angst.\cite{McCarthy2015, Sundhedsstyrelsen2015} 

Et led i rehabiliteringen er ligeledes, at patienter opnår viden indenfor sygdomshåndtering, der omhandler kendskab til og forebyggelse af sygdommen, livsstilsændringer samt håndtering af eksacerbationer.\cite{McCarthy2015,Sundhedsstyrelsen2015} Her fokuseres blandt andet på de gavnlige effekter ved tobaksophør og regelmæssig fysisk aktivitet, samt hvornår og hvordan eventuel medicin skal indtages. Patienten vil yderligere blive introduceret til energibesparende strategier og vejrtrækningsøvelser.\cite{McCarthy2015,Sundhedsstyrelsen2015}    

Gennem studier er det oplyst, at ikke alle patienter er i stand til at opretholde resultaterne, og deres fysiske tilstand falder tilbage til niveauet før rehabiliteringsforløbet \cite{Egan2012,Beachamp2013,Zanaboni2017,Ringbaek2008}. 
Årsagerne til dette tilbagefald kan blandt andet være som følge af, at rehabiliteringen ikke er med til at gøre patienter mere aktive i hjemmet efter afsluttet forløb, da de falder tilbage til deres vante rutiner \cite{Egan2012}. Ligeledes ses det hos patienter, der fortsat træner, at intensiteten og hyppigheden af træningen falder \cite{Ringbaek2008}. 

Der er forskellige metoder til at opretholde patienters resultater efter afsluttet rehabiliteringsforløb. Dansk Selskab for Almen Medicin (DSAM) anbefaler KOL-patienter at følge et vedligeholdelsesprogram bestående af fire til fem træningssessioner om ugen efter afsluttet rehabiliteringsforløb. Derudover tilbydes  KOL-patienter at deltage i fælles træningssessioner et par gange ugentligt til månedligt.\cite{dsam2016}
Fordele ved gruppeaktiviteter er, at KOL-patienter kan undgå social isolation samtidig med, at de lærer af hinandens erfaringer i forhold til, hvordan de hver især håndterer sygdommen. Herved kan sociale fællesskaber være en medhjælpende faktor til vedligeholdelse af effekten ved rehabiliteringen.\cite{dsam2016} I Danmark er der forskellige fællesskaber, hvor KOL-patienter har mulighed for at danne træningsgrupper og afholde arrangementer.\cite{Sundhedsstyrelsen2015} Hertil har Lungeforeningen i Danmark forskellige lokalafdelinger, hvor der et par gange årligt afholdes arrangementer for patienter samt pårørende. Ligeledes er der forskellige netværksgrupper, hvor patienter kan være sociale til træningssessioner og andre arrangementer.\cite{Lungeforeningen2016}

*** HER MANGLER EN OVERGANG ***

\section{Applikationer med relation til KOL-patienter}
På nuværende tidspunkt findes forskellige danske applikationer (app), herunder KOL-guidelines, Kronika og HomeRehab, der kan anvendes af  sundhedspersonale og patienter som et hjælpemiddel til forståelsen samt rehabiliteringen af KOL.\cite{KOLguidelines2012,Kronika2014,HealthcareDenmark2017} 

KOL-guidelines er udviklet af Boehringer Ingelheim, der er baseret på guidelines fra Dansk Lungemedicinsk Selskab (DLS), og anvendes som led i at klassificere sværhedsgraden af KOL ud fra antal af eksacerbationer, MRC og CAT. Ud fra sværhedsgraden kommer appen med anbefalede behandlingsmuligheder med henblik på fysisk træning, farmakologisk behandling, vaccination og essentielle elementer i symptomlindring.\cite{KOLguidelines2012}    

Kronika, som er udviklet af Region Hovedstaden giver generel viden omkring forskellige kroniske sygdomme heriblandt hjertekrampe, hjertesvigt, KOL, tryksår og type 2 diabetes. I relation til KOL giver appen viden omkring sygdommen og vejledninger om vejrtrækningsteknikker samt stillinger, der reducerer åndenød. Ligeledes fremhæves generelle anbefalinger som fysisk aktivitet samt,  hvilke sygdomsmarkører, der fremkommer ved optakt til en eventuel sygdomsforværrelse.\cite{Kronika2014}    

De ovennævnte apps vil primært give sundhedspersonalet et overblik over, hvilke anbefalinger, der gives til KOL-patienter. Disse fungerer dermed som opslagsværk for sygdommen. I relation til opretholdelse af resultaterne fra rehabiliteringen, anses de ikke til at have en gavnlig effekt. 
Dette skyldes blandt andet, at de nævnte apps ikke er udviklet til patienter, men for det varetagende sundhedspersonale.\cite{Kronika2014} \fxnote{Ud fra tidligere nævnte faktorer for vedligeholdelse af rehabiliteringsresultater i afsnit XX, ville en app blandt andet skulle være i stand til at motivere patienten til at være fysisk aktiv i hverdagen, hvilket ikke fremkommer hos de førnævnte apps.} 

HomeRehab, udviklet af Firmaet Aidcube, kan anvendes under og efter rehabiliteringsforløbet. HomeRehab har til formål at gøre KOL-patienter i stand til at tage varetage sig selv ved at opretholde effekterne af rehabiliteringen gennem motivering til daglig træning. Data fra HomeRehab appen hjælper også sundhedspersonale, der kan tilgå data via en webportal, med at identificere tegn på sygdomsforværring, hvilket anvendes til at reducere risikoen for hospitalsindlæggelse. HomeRehab testes på nuværende tidspunkt i samarbejde med blandt andet Hvidovre Hospital, Frederiksberg Hospital og Silkeborg Kommune. 
I testperioden bliver en behandlingsgruppe introduceret i anvendelse af HomeRehab i et 7 ugers rehabiliteringsforløb, hvor de efterfølgende anvender appen hjemme de efterfølgende 6 måneder.\cite{HealthcareDenmark2017}

På nuværende tidspunkt er der ikke udgivet forskningsartikler omhandlende effekterne fra anvendelsen af HomeRehab appen, dog fremgår resultater fra prøveperioden i Silkeborg Kommune på Aidcubes hjemmeside. Dertil oplyses, at 68 \% af KOL-patienterne i behandlingsgruppen opretholder deres daglige aktivitet og har medført 10 \% reduktion i KOL-relaterede hospitalsindlæggelser.\cite{AidCube2017}
