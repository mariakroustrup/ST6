\section{Rehabilitering af KOL}
Da KOL er kronisk lungesygdom, hvorved den tabte lungefunktion ikke kan genoprettes, tilbydes KOL-patienter rehabilitering med henblik på at mindske deres symptomer. 

I Danmark henvises KOL-patienter til rehabilitering fra praktiserende læge eller hospital. Rehabilitering forløber typisk over $8$ ugers periode på et sundhedscenter eller på et hospital. KOL-patienter møder til træning $1$-$2$ gange om ugen, de resterende dage vil patienten kunne udføre de fremviste øvelser hjemme. \cite{McCarthy2015,Frausing2011b} 

Individuel rehabilitering anses som værende fundamental for KOL-patienter, hvor forløbet tilpasses i forhold til patienternes behov med henblik på at opnå det bedste udbytte af rehabiliteringen. \cite{McCarthy2015,Habraken2011,Sundhedsstyrelsen2015} Ligeledes vurderes rehabiliteringen på baggrund af graden af KOL, da KOL fremkommer i flere grader samt med varierende progression \cite{McCarthy2015}. 

Rehabiliteringen kan give patienter bedre mulighed for deltagelse i hverdagen, såfremt patienters tilstand tillader det. \cite{McCarthy2015,Habraken2011, Sundhedsstyrelsen2015} Opfølgninger kan foretages efter rehabiliteringsforløb er afsluttet, for at undersøge om patienter opretholder de gavnlige effekter. \cite{Frausing2011b}


\subsection{Rehabiliteringsforløb}
Rehabiliteringsforløbet fokuserer på tobaksafvænning, fysisk træning, kendskab til sygdommen samt ernæringsvejledning. \cite{McCarthy2015,Habraken2011,Sundhedsstyrelsen2015} 

Tobaksafvænning er, som beskrevet i \autoref{sec:behandling}, et relevant element i forhold til at begrænse udviklingen af sygdommen og bevare mest mulig lungefunktion. Den fysiske træning, der udføres under rehabiliteringen, medvirker til, at patienter kan opnå et bedre udbytte af den resterende lungefunktion, samt opnå et bedre fysisk funktionsniveau. \cite{Sundhedsstyrelsen2015}
Træningen vil ligeledes modvirke eventuelle følger ved KOL, da fysisk træning øger muskelfunktionen samt udsætter træthed, hvilket medfører forøget aktivitetstolerance \cite{McCarthy2015}. Dog kan den fysiske træning resultere i åndenød hos KOL-patienter, der kan forstærkes, hvis patienter påvirkes af angst som følge af åndenød. Dette kan betyde, at KOL-patienter afholder sig fra fysisk træning på grund af frygten for angst. \cite{McCarthy2015, Sundhedsstyrelsen2015} 

Et led i rehabiliteringen er ligeledes, at patienter opnår viden indenfor sygdomshåndtering, der omhandler kendskab til og forebyggelse af sygdommen, livsstilsændringer samt håndtering af eksacerbationer.\cite{McCarthy2015,Sundhedsstyrelsen2015} Her fokuseres blandt andet på de gavnlige effekter ved tobakophør og regelmæssig fysisk aktivitet, samt hvornår og hvordan eventuel medicin skal indtages. Patienten vil yderligere blive introduceret til energibesparende strategier og vejrtrækningsøvelser. \cite{McCarthy2015,Sundhedsstyrelsen2015}    

Det ses dog fra flere studier, at patienter ikke er i stand til at opretholde resultaterne, og deres fysiske tilstand falder mod tilstanden som den var før rehabiliteringsprogrammet. \cite{Egan2012, Beauchamp2013, Zanaboni2017,Ringbaek2008} 
KOL-patienter kan opnå gode resultater fra 7-8 ugers rehabiliteringsforløb, dog ses det efter ét år, at nogle af patienterne ikke er i stand til at opretholde resultaterne. \cite{Egan2012, Beauchamp2013} Årsagerne til dette fald kan blandt andet være som følge af, at rehabiliteringen ikke er med til at gøre patienterne mere aktive i hjemmet efter afslutning af forløbet, da de falder tilbage til deres vante adfærd og rutine. \cite{Egan2012} Ligeledes ses det hos patienter, der fortsat træner, at intensiteten og hyppigheden af træningen falder. \cite{Ringbaek2008} 

Der er forskellige måder aktivt at hjælpe med at opretholde patienternes resultat efter et afsluttet rehabiliteringsprogram. Dansk Selskab for Almen Medicin (DSAM) anbefaler KOL-patienter at følge et vedligeholdelsesprogram bestående af 4-5 træningssessioner om ugen efter afslutning af RF. Ligeledes kan KOL-patienter deltage i fælles træningssessioner et par gange ugentligt til månedligt. \cite{dsam2016}
Fordele ved gruppeaktiviteter er, at KOL-patienter kan undgå social isolation samtidig med, at de kan lære af hinandens erfaringer i forhold til, hvordan de hver især håndterer sygdommen. Herved kan sociale fællesskaber være en medhjælpende faktor til vedligeholdelse af effekten fra rehabiliteringen.  \cite{dsam2016} 
I Danmark er der forskellige fællesskaber, hvor patienter har mulighed for at danne træningsgrupper og afholde arrangementer. \cite{Sundhedsstryrrelsen2015} Hertil har Lungeforeningen i Danmark forskellige lokalafdelinger, hvor der et par gange om året afholdes arrangementer for patienter og pårørende. Ligeledes er der forskellige netværksgrupper, hvor patienter kan komme sammen til træningssessioner og andre arrangementer. \cite{Lungeforeningen2016}

\subsection{Apps med relation til KOL-patienter}
På nuværende tidspunkt findes forskellige danske apps, der kan anvendes af sundhedspersonale som et hjælpemiddel i behandlingen af KOL. Af disse findes følgende af offentligt tilgængelige apps. 

Boehringer Ingelheim har udviklet appen, ‘KOL-guidelines’, der er baseret på guidelines fra Dansk Lungemedicinsk Selskab (DLS), og denne kan anvendes som led i at klassificere sværhedsgraden af KOL ud fra antal af eksacerbationer, CAT og MRC. Ud fra klassificeringen kommer app’en med anbefalet behandlingsmulighed, med henblik på fysisk træning, farmakologisk behandling, vaccination og essentielle elementer i symptomlindring.    
‘Kronika’, som er udviklet af Region Hovedstaden, er en app, der giver generel viden omkring forskellige kroniske sygdomme, heriblandt hjertekrampe, hjertesvigt, KOL, tryksår og type-2 diabetes. I relation til KOL giver app’en viden omkring sygdommen og vejledninger i, hvordan vejrtrækningsteknik, samt stillinger, der reducerer åndenød. Ligeledes fremhæves generelle anbefalinger som fysisk aktivitet, og yderligere oplyses hvilke sygdomsmarkører, der fremkommer ved optakt til en eventuel sygdomsforværrelse.    

Informationen givet fra de de ovennævnte apps, vil primært give en sundhedspersonale et overordnet blik over hvilke anbefalinger, der gives til KOL patienter. De to apps fungerer dermed som opslagsværk KOL, men i relation til vedligeholdelse af effekterne fra rehabiliteringen, anses de til ikke at have en gavnlig effekt. 
Dette skyldes bl.a. at de nævnte apps ikke er udviklet til patienterne, men for det varetagende sundhedspersonale. Ud fra tidligere nævnte faktorer for vedligeholdelse af rehabiliteringsresultater i afsnit XX, ville en app blandt andet skulle være i stand til at motivere patienten til at være fysisk aktiv i hverdagen, hvilket ikke fremkommer hos de førnævnte apps. 

Firmaet Aidcube har udviklet en app, ‘HomeRehab’, der vil kunne anvendes i og efter et rehabiliteringsforløb. HomeRehab har til formål at gøre KOL-patienter i stand til at tage vare for sig selv, ved at opretholde effekterne af rehabiliteringen gennem motivering til daglig motion. Data fra HomeRehab appen hjælper ligeledes også sundhedspersonale, der kan tilgå data via en web-portal, med at identificere tegn på sygdomsforværring, hvilket anvendes til at reducere risikoen for hospitalsindlæggelse. \cite{HealthcareDenmark2017}

HomeRehab testes på nuværende tidspunkt i samarbejde med blandt andet Hvidovre Hospital, Frederiksberg Hospital og Silkeborg Kommune. 
I testperioden bliver en behandlingsgruppe introduceret i anvendelse af HomeRehab i et 7 ugers rehabiliteringsforløb, hvor de efterfølgende anvender appen hjemme de efterfølgende 6 måneder. \cite{HealthcareDenmark2017}

På nuværende tidspunkt er der ikke udgivet forskningsartikler omhandlende effekterne fra anvendelsen af HomeRehab appen, dog fremgår resultater fra prøveperioden i Silkeborg Kommune på Aidcubes hjemmeside. Disse oplyser at 68 \% af KOL-patienterne i behandlingsgruppen, opretholder deres daglige aktivitet, og har medført en 10 \% reduktion i KOL-relaterede hospitalsindlæggelser. \cite{AidCube2017}
