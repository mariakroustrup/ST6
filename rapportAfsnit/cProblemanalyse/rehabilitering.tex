\section{Rehabilitering}
Da KOL er kronisk lungesygdom, hvorved den tabte lungefunktion ikke kan genoprettes, tilbydes KOL-patienter rehabilitering med henblik på at mindske deres symptomer. 

I Danmark henvises KOL-patienter til rehabilitering fra praktiserende læge eller hospital. Rehabilitering forløber typisk over $8$ ugers periode på et sundhedscenter eller på et hospital. KOL-patienter møder til træning $1$-$2$ gange om ugen, de resterende dage vil patienten kunne udføre de fremviste øvelser hjemme. \cite{McCarthy2015,Frausing2011b} 

Individuel rehabilitering anses som værende fundamental for KOL-patienter, hvor forløbet tilpasses i forhold til patienternes behov med henblik på at opnå det bedste udbytte af rehabiliteringen. \cite{McCarthy2015,Habraken2011,Sundhedsstyrelsen2015} Ligeledes vurderes rehabiliteringen på baggrund af graden af KOL, da KOL fremkommer i flere grader samt med varierende progression \cite{McCarthy2015}. 

Rehabiliteringen kan give patienter bedre mulighed for deltagelse i hverdagen, såfremt patienters tilstand tillader det. \cite{McCarthy2015,Habraken2011, Sundhedsstyrelsen2015} Opfølgninger kan foretages efter rehabiliteringsforløb er afsluttet, for at undersøge om patienter opretholder de gavnlige effekter. \cite{Frausing2011b}


\subsection{Rehabiliteringsforløbet}
Rehabiliteringsforløbet fokuserer på tobaksafvænning, fysisk træning, kendskab til sygdommen samt ernæringsvejledning. \cite{McCarthy2015,Habraken2011,Sundhedsstyrelsen2015} 

Tobaksafvænning er, som beskrevet i \autoref{sec:behandling}, et relevant element i forhold til at begrænse udviklingen af sygdommen og bevare mest mulig lungefunktion. Den fysiske træning, der udføres under rehabiliteringen, medvirker til, at patienter kan opnå et bedre udbytte af den resterende lungefunktion, samt opnå et bedre fysisk funktionsniveau. \cite{Sundhedsstyrelsen2015}
Træningen vil ligeledes modvirke eventuelle følger ved KOL, da fysisk træning øger muskelfunktionen samt udsætter træthed, hvilket medfører forøget aktivitetstolerance \cite{McCarthy2015}. Dog kan den fysiske træning resultere i åndenød hos KOL-patienter, der kan forstærkes, hvis patienter påvirkes af angst som følge af åndenød. Dette kan betyde, at KOL-patienter afholder sig fra fysisk træning på grund af frygten for angst. \cite{McCarthy2015, Sundhedsstyrelsen2015} 

Et led i rehabiliteringen er ligeledes, at patienter opnår viden indenfor sygdomshåndtering, der omhandler kendskab til og forebyggelse af sygdommen, livsstilsændringer samt håndtering af eksacerbationer.\cite{McCarthy2015,Sundhedsstyrelsen2015} Her fokuseres blandt andet på de gavnlige effekter ved tobakophør og regelmæssig fysisk aktivitet, samt hvornår og hvordan eventuel medicin skal indtages. Patienten vil yderligere blive introduceret til energibesparende strategier og vejrtrækningsøvelser. \cite{McCarthy2015,Sundhedsstyrelsen2015}    

