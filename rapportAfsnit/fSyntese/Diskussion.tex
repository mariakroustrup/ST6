\section{Diskussion}
I dette kapitel diskuteres væsentlige problemstillinger forbundet med problemformuleringen. Formålet med projektet er at udvikle en app, der kan vejlede og motivere KOL-patienter til regelmæssig træning. Dette indebærer, hvad der er gjort for at besvare problemformuleringen samt mulige forbedringer i forhold til app'en, hvorfor det er valgt at diskutere kravspecifikationer, design og implementering samt, hvilke mulige ændringer og tilpasninger, der skal foretage for at kunne anvende  app’en i praksis.

\subsection{Kravsspecifikationer}
De opstillede funktionelle krav til app’en er testet og alle opfyldt. Det kan dog diskuteres, hvorvidt de opstillede krav skal genovervejes, hertil kunne det blandt andet diskuteres, hvorvidt et fastlagt tidspunktet for notifikation vil være hensigtsmæssigt for brugeren. Det formodes, at brugerne af app’en træner på forskellige tidspunkter i løbet af dagen, hvorved en notifikation kan virke forstyrrende for brugeren, hvis der allerede er udført en træning tidligere på dagen. En løsning på dette kunne være, at notifikationen annulleres, hvis brugeren har trænet inden klokken 15 eller, at notifikationen kommer 24 timer efter sidste udførte træning. Derudover er det en mulighed, at notifikationen tilpasses brugerens kategorisering og helbredstilstand, da de måske ikke er i stand til daglig motion, hvorfor notifikationen kan virke demotiverende og for eksempel først skal forekomme efter 48 timer fra sidste udførte træning. Yderligere kan det diskuteres, om den motiverende faktor kan optimeres ved brugen af notifikation på andre måder. For eksempel ved at give feedback i form af ros, hvis brugeren har udført en træning eller informere brugeren om, hvor mange kilometer brugeren mangler for at opnå en belønning. Dog kan for mange notifikationer have en modsatrettet effekt, hvor brugeren fravælger brug af app’en. 

Det har ikke været muligt at teste non-funktionelle krav, da app’en kun er en prototype. Men da app’en er designet og derved også implementeret ud fra gestaltlovene omfattende opsætning af grænseflader samt egenskaber, der har betydning for brugervenligheden, antages det, at de non-funktionelle krav er opfyldt. Dette er antaget på baggrund af, at app'ens forskellige grænseflader tager udgangspunkt i samme layout samt indeholder få knapper, der gentages løbende i app’en, som for eksempel videreknappen. 

\subsection{Design}
I afsnittet for design er det valgt at afgrænse i forhold til træning. Projektets primære formål er ikke at udarbejde et træningsprogram, men at udvikle en app, hvorpå et træningsprogram senere kan inkorporeres. Afgrænsningen til konditionstræning er valgt, da målinger, der vil foretages ved konditionstræning, herunder tid og afstand er et krav og derfor skal implementeres og testes. Derudover skal det diskuteres, om andre målinger kan være mere hensigtsmæssige at implementere ved valg af styrketræning og vejrtrækningsøvelser, hvor afstand ikke anses som værende en essentiel måling. Det kunne for eksempel være mere hensigtsmæssig at implementere antallet af gentagelser. 

Det er valgt at designe tilpasning af træningsniveauet ud fra en beslutningstabel, som medregner parametre, såsom kategorisering, daglig helbredstilstand og tidligere evalueringer, der afhænger af tidligere angivne helbredstilstande fra samme træningstype. Dette er gjort for at tage højde for daglige variationer samt brugerens evne til at vurdere sin daglige helbredstilstand i forhold til, hvad KOL-patienter fysisk kan yde. De tidligere evalueringer, der afhænger af, at brugeren har valgt samme helbredstilstand og træningstype, er ikke afhængig af tid, hvilket vil sige, at denne evaluering kan være forældet, hvis brugerens opfattelse af deres helbredstilstand ændres. En bruger, der tidligere har angivet en daglig helbredstilstand til moderat vil for eksempel, hvis det generelle helbred forværres, senere angive denne helbredstilstand til god, da opfattelsen af helbredstilstanden er ændret. Den tilhørende evaluering vil derfor ikke være sigende. 

Det har ikke været muligt at finde litteratur eller teste i, hvilken grad parametrene har indflydelse på KOL-patienters helbred og fysiske egenskaber samt, hvordan den enkelte opfatter sin tilstand, hvis den ændres, og derved ikke angiver en sigende evaluering. En mulig løsning til disse problemstillinger kan være at anvende bayesian læring, som kan forbedre tilpasningen af træningsniveau over tid på baggrund af de angivne parametre og derved yderligere specialisere træningsniveauet til den enkelte bruger. Dette vil ligeledes tage højde for den ændring i opfattelse af helbredstilstand, der muligvis kan forekomme.

Et af projektets formål er at motivere brugeren til at udføre regelmæssig motion, hvilket er forsøgt opfyldt ved, at brugeren kan opnå belønninger. I app’en fremgår det ikke, hvad der skal til for at opnå en belønning, og det kan diskuteres, hvorvidt dette skal være en mulighed for at øge motivation yderligere. En ulempe ved dette kan være, at brugeren bliver for konkurrenceminded og derved fravælger træningstyper samt overanstrenger sig for at opnå en belønning. Belønningerne opnås på samme måde uafhængig af kategorisering, hvilket vil medvirker til, at en bruger som er kategoriseret som D, vil skulle træne lige så meget som en, der er kategoriseret i A. Det kan diskuteres, hvorvidt det vil virke demotiverende for brugerne, hvis belønningerne opleves for lette eller svære, da brugerne ikke kan præstere på samme vis.  Inddelingen af belønninger vil derfor være mest hensigtsmæssigt at lave på baggrund af kategoriseringer. 

Yderligere er det forsøgt at øge brugerens  motivation ved at designe social interaktion, hvor brugere har mulighed for at se hinandens belønninger. Her vil der kunne opstå uligheder, hvis brugere med forskellige kategoriseringer vælger at følge hinanden, da belønningerne, som tidligere nævnt, ikke gives på baggrund af kategoriseringen. Det kan diskuteres, hvorvidt dette vil virke demotiverende eller motiverende for brugerne. Brugere med en kategorisering i A vil for eksempel kunne blive demotiverede, da niveauet ikke er højt nok eller føle sig mere motiverede, idet de opnår mange belønninger i forhold til andre venner. Modsat vil brugere, der kategoriseres i D, have svært ved at opnå belønninger i forhold til andre venner og måske anstrenge sig uhensigtsmæssigt for at opnå belønninger. 

I forbindelse med social interaktion er app’en ikke designet, så brugeren kan fravælge, at andre brugere kan følge den. Dette kan muligvis være krænkende  for nogle brugere, hvorved de kan fravælge brug af app’en. En løsning på dette kan være, at belønningerne er usynlige indtil brugeren har accepteret, at en anden bruger må følge den. Modsat vil dette kunne demotivere nogle brugere, hvis de ikke tillader at andre brugere kan tilgå deres resultater, da konkurrenceelementet vil kunne svækkes. For eksempel vil brugeren ikke blive påvirket på samme måde som, hvis andre brugere ikke kan se dens belønninger.

\subsection{Implementering}
Det er valgt at implementere en database til lagring af brugerdata. Databasen blev 
implementeret på en lokal server. En fordel ved at gøre dette er, at det simulerer en endelig database og foretrækkes at anvende under en udviklingsfase, da eventuelle fejl er lettere at rette. Derudover begrænser en lokal server på nuværende tidspunkt, at app’en ikke kan anvendes på en smartphone, som ifølge det opstillede non-funktionelle krav er hensigten. Det forventes dog, at app'en kan anvendes på en smartphone, hvis databasen implementeres på en ekstern server, idet app’en på nuværende tidspunkt er simuleret i en android-emulator. For at app’en endelig kan implementeres skal der foretages ændringer i forhold til at  implementere databasen på en ekstern server.
En ulempe ved en ekstern server er, at det er mere omstændig at rette eventuelle fejl sammenlignet med en lokal server. 

Da det er valgt at implementere databasen på en lokal server, er det ikke valgt at kryptere data, da dette ikke ses lige så nødvendigt som, hvis app’en var implementeret på en ekstern server. Det kan dog diskuteres, hvorvidt data skal krypteres af sikkerhedsmæssige årsager. Det er valgt at anvende medlemsID’er som identifikation fremfor personnumre i databasen af samme årsag. Det kan dog diskuteres, hvorvidt den enkelte vil anse data som værende personfølsomme, hvorfor disse muligvis burde krypteres. 

På baggrund af designafsnittet blev der valgt at implementere en timer til at måle tiden under træning. Det er valgt at implementere denne som en optælling, da det ønskes, at brugere skal have mulighed for at fortsætte træningen efter den anbefalede træningstid er opnået. Det kan diskuteres, om dette er den mest hensigtsmæssige måde, idet tanken om at implementere et anbefalet træningsniveau er at sikre, at brugere ikke underpræsterer eller overpræsterer. På nuværende tidspunkt er der ikke implementeret en form for feedback, når brugeren har opnået det anbefalede træningsniveau, hvormed brugeren selv skal holde øje med, om dette er opnået. En løsning på dette kunne være at implementere timeren som en nedtælling eller at indføre lyd, der gør brugeren opmærksom, når det anbefalede træningsniveau er opnået. 