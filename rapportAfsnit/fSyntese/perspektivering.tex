\section{Perspektivering}
Da denne app er udviklet som en prototype, er der flere ting, som skal overvejes inden en endelig implementering. På nuværende tidspunkt er serveren lokal, hvorved databasen kun kan tilgås via én computer. For at sundhedspersonalet fra flere rehabiliteringscentre i fremtiden skal kunne tilgå databasen, skal det overvejes at implementere denne som en tilgængelig server. Derudover skal det overvejes om sundhedspersonalet via app’en eller ved udvikling af en ny app, skal have mulighed for at oprette nye brugere i databasen samt kunne tilgå brugernes resultater via denne.  

I forhold til tilpasning af træningsniveau skal der undersøges flere studier i forhold til faktorer, der påvirker KOL-patienters helbredstilstand. Dette kunne eksempelvis være at inddrage komorbiditeter, blodtryk, vejrudsigter, om patienterne er blue bloater eller pink puffer. Dette ville medvirke til, at algoritmen for udregning af anbefalet træningsniveau vil kunne tilpasses den enkelte KOL-patient bedre end på nuværende tidspunkt. Det anbefalede træningsniveau illustreres ved tid, hvorved det skal overvejes, om det vil være mere hensigtsmæssigt at inddrage distance i stedet, eller om begge anbefalinger skal fremgå. I forhold til træningsformer og -typer skal det undersøges i separat studie, hvilke der vil være mest hensigtsmæssig at udføre for patienter med KOL. Dertil skal det overvejes, om sundhedspersonalet skal have mulighed for at tilføje ekstra træningsformer eller øvelser, som patienterne har udført og har kendskab til i forbindelse med rehabiliteringsforløbet.

Til træningen kunne det overvejes at inddrage flere enheder til monitorering af træningen. Dette kunne for eksempel være eksterne enheder som pulsur og iltmåling til biologiske målinger, der kan være med til at vejlede patienten i forhold til at opnå mest ud af træningen samt advare patienten ved overanstrengelse. Derudover kunne det overvejes at gøre det muligt at tilkoble træningsmaskiner, såsom sofacykel, kondicykel og løbebånd, så patienter med adgang til disse kan anvende disse sammen med app’en. Dette vil dog kræve, at træningsmaskinerne skal kunne kommunikere med app’en, for eksempel via bluetooth. 

Resultaterne, opnået ved træning, er på nuværende tidspunkt vist som virtuelle belønninger. Dertil bør det overvejes, om grafisk udvikling eller anden visning vil være mere motiverende for  brugeren. Ved grafisk udvikling vil sundhedspersonalet også få et visuelt indblik i, hvordan de enkelte brugeres udvikling forløber. Hertil vil det også forventes, at sundhedspersonalet vil kunne motivere og give besked, hvis patienter har udviklet sig samt, hvis patienterne har været inaktiv i en længere periode. 

I forhold til venneliste skal opsætningen af denne overvejes. Det kunne være muligt at rangordne brugere, der følges, efter opnåede belønninger med henblik på motivering. Derudover skal det overvejes, om brugeren skal have notifikationer, når andre venner træner og har opnået belønninger. 