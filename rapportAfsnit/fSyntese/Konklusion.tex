\section{Konklusion}
KOL er en kronisk sygdom, hvorfor det ikke er muligt at helbrede patienter med sygdommen. Derfor tilbydes patienter med KOL at deltage i et rehabiliteringsforløb med henblik på at reducere symptomer forbundet med sygdommen. Dette indebærer tobaksafvænning, fysisk træning, kendskab til sygdommen samt ernæringsvejledning. Studier viser, at resultaterne fra deltagelse på rehabiliteringsforløb har en positiv effekt, dog er KOL-patienterne ikke i stand til at opretholde resultaterne 6-12 måneder efter endt rehabiliteringsforløb. Sociale fællesskaber kan have en positiv virkning på at KOL-patienter kan opretholde effekten af resultaterne hjemme. Derudover anvendes flere forskellige telerehabiliteringsteknologier, herunder app’s, der har til formål at hjælpe patienter med at opretholde opnåede effekter uden for sundhedspleje faciliteter. 

Det er i projekt derfor valgt at udvikle en app til at vejlede og motivere KOL-patienter til hjemmetræning i forlængelse af rehabiliteringsforløb med henblik på at mindske symptomer forbundet med KOL.

Dertil er der udarbejdet en app, der tager højde for daglige variationer ved at tilpasse træningsniveauet ud fra parametre, såsom ABCD-kategorisering, daglig helbredstilstand samt tidligere evalueringer af træninger. Ud fra disse parametre anbefales et træningsniveau, som er en vejledning for brugeren. Under træningen anvendes en timer, så brugeren kan følge med i, hvornår det anbefalede træningsniveau er opnået. For at motivere brugere kan der opnås virtuelle belønninger på baggrund af den udførte træning. Derudover har brugere mulighed for at se andre brugeres virtuelle belønninger via en venneliste. Denne venneliste giver ligeledes mulighed for social interaktion, da brugere kan vælge at tilføje venner til vennelisten. Yderligere gøres brugeren opmærksom på træning hver dag ved en gentagende notifikation. Dette er også med henblik på at motivere KOL-patienterne til at dyrke regelmæssig motion.  

De centrale elementer, der skal udgøre app’en, herunder tilpasning af træningsniveau og motivering ved blandt andet social interaktion, er testet og er på baggrund af denne test opfyldt. Det har dog ikke været muligt at teste i, hvilket omfang app’en tilpasser træningsniveauet til den enkelte samt i hvor høj grad den motiverer brugeren til regelmæssig træning samt, hvilken effekt den sociale interaktion har på motivationen. App’en giver dog muligheden for at tilpasse et træningsniveauet samt motivere ved at opnå belønninger og social interaktion. Yderligere studier skal derfor udføres for at kunne bekræfte eller afkræfte, hvilken effekt brug af app’en har på KOL-patienter i forhold til opretholdelse af resultater efter endt rehabilieringsforløb herunder, hvorvidt dette vil mindske symptomer forbundet med KOL. 