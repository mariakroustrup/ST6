\section{Konklusion}
KOL er en kronisk sygdom, hvorfor det ikke er muligt at helbrede patienter. Derfor tilbydes patienter med KOL at deltage i et rehabiliteringsforløb med henblik på at lindre symptomer forbundet med sygdommen. Dette indebærer tobaksafvænning, fysisk træning, kendskab til sygdommen samt ernæringsvejledning. Studier viser, at resultaterne fra deltagelse på rehabiliteringsforløb har en positiv effekt, dog er KOL-patienterne ikke i stand til at opretholde resultaterne ét halvt til ét helt år efter endt rehabiliteringsforløb. Sociale fællesskaber kan have en positiv virkning på, at nogle KOL-patienter kan opretholde effekten af resultaterne hjemme. Derudover anvendes flere forskellige telerehabiliteringsteknologier, herunder app’s, der har til formål at hjælpe patienter med at opretholde opnåede effekter udenfor sundhedsvæsnets faciliteter. 

Det er på baggrund af dette valgt at udvikle en app til at vejlede og motivere KOL-patienter til hjemmetræning i forlængelse af rehabiliteringsforløb med henblik på at lindre symptomer forbundet med KOL.

Den udarbejdede app tager højde for daglige variationer ved at tilpasse træningsniveauet ud fra parametre, såsom ABCD-kategorisering, daglig helbredstilstand samt tidligere evaluering af træning. Ud fra disse parametre anbefales et træningsniveau, som er en vejledning for brugeren. Under træningen anvendes en timer, så brugeren kan følge med i, hvornår den anbefalede træningstid er opnået, dertil afspilles en lydfil for at gøre brugere opmærksom på dette. For at motivere brugere kan brugeren se sin udvikling grafisk, samt opnå virtuelle belønninger på baggrund af udført træning. Derudover har brugere mulighed for at se andre brugeres virtuelle belønninger via en venneliste. Denne venneliste giver ligeledes mulighed for social interaktion, da brugere kan vælge at tilføje venner til vennelisten. Yderligere gøres brugeren opmærksom på træning hver dag ved en gentagende notifikation, med henblik på at informere samt motivere KOL-patienterne til at dyrke regelmæssig motion.  

De centrale elementer, der indgår i app’en, herunder tilpasning af træningsniveau og motivering ved blandt andet social interaktion, er testet og på baggrund af disse tests opfyldt. Det har dog ikke været muligt at teste i, hvilket omfang app'en tilpasser træningsniveauet til den enkelte brugere samt i, hvor høj grad app'en motiverer til regelmæssig træning. Dertil har det ligeledes ikke været muligt at teste den sociale interaktions virkning på KOL-patienter. App’en giver dog muligheden for at tilpasse et træningsniveauet samt motivere ved at opnå belønninger og social interaktion. 

Da denne app er en prototype, skal der foretages ændringer og tilføjelser for at muliggøre implementering af denne i praksis. Dette indebærer blandt andet, at der implementeres øvelser passende til KOL-patienter samt anbefalede træningsniveauer, som er realistiske i forhold til, hvad KOL-patienter med forskellige ABCD-kategoriseringer og helbredstilstande fysisk kan holde til. Dertil skal træningsformer og -typer ligeledes tilpasses efter øvelser og træninger, der foretages i forbindelse med rehabiliteringsforløbet, således det sikres, at KOL-patienter har kendskab til de træningsformer samt -typer, der implementeres i app’en. 

Yderligere studier skal undersøges med henblik på at implementere træningsformer, -typer og anbefalede træningsniveauer, som er passende til KOL-patienter. Ligeledes skal studier udføres for at kunne bekræfte, hvilken effekt brug af app’en har på KOL-patienter i forhold til opretholdelse af resultater efter endt rehabilieringsforløb herunder, hvorvidt dette vil lindre symptomer forbundet med KOL. 