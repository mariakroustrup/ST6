\chapter{Bilag}
\section{Lungernes anatomi og fysiologi} \label{BilagA}
Kroppens to lunger er omgivet af thorax og ligger i thoraxhulen. Thorax består af sternum og de ydre og indre interkostal muskler, herunder thorakal hvirvlerne, ribbenene samt musklerne mellem ribbenene. Halsmusklerne udgør loftet og diafragma udgør gulvet af sternum. De forskellige muskler og diafragma benyttes under ventilation. \citep{Martini2012, Sand2008}

\subsection{Ventilation}
Ventilation beskriver transport af luft frem og tilbage mellem atmosfæren og lungealveolerne. Luften bevæger sig fra områder med højt tryk til områder med lavt tryk. Det atmosfæriske tryk ændres ikke under normale omstændigheder, hvorfor det er variationer i trykket i alveolerne, der sørger for transporten af luft. Trykket skabes ved at lungerne udvides og presses sammen, herved bliver alveoletrykket lavere eller højere end det atmosfæriske tryk, hvilket medfører inspiration eller eksspiration. \citep{Martini2012, Sand2008}

\subsubsection{Inspiration}
Inden inspiration er alle interkostal musklerne afslappede og alveoletrykket er det samme som det atmosfæriske tryk, hvilket betyder at der ikke strømmer luft gennem luftvejene. Ved inspiration udvides thoraxhulen, som medvirker til at trykket inde i lungen falder, hvormed  lungerne udvider sig. Kontraktion af interkostal musklerne øger volumen i både bredden og dybden. Diafragma trækker sig sammen medfører at thoraxhulens volumen øges. Halsmusklerne hæver ribbenene, hvilket yderligere medvirker til øget volumen af thoraxhulen. Når inspirationen er afsluttet afslappes interkostal musklerne igen.\citep{Martini2012, Sand2008}

\subsubsection{Eksspiration}
Eksspiration sker når alveoletrykket inde i lungerne er større end det atmosfæriske tryk uden for. For at udligne trykket presses diafragma op mod sternum,interkostal musklerne trækker sig sammen, hvormed ribbene trækkes nedad. Dette medfører at lungernes volumen mindskes. Eksspirationen fortsættes indtil trykforskellen mellem det atmosfæriske tryk og alveoletrykket er udlignet. \citep{Martini2012, Sand2008}

\subsection{Respiration}
Respirationssystemet består af et øvre og et nedre. Det øvre respirationssystem består af næsen, næsehulen, paranasal sinuses fxnote{er en gruppe af fire parrede luftfyldte rum, der omgiver næsehulen, herunder kæbehulerne, frontale bihuler, ethmoidale bihuler, sphenoidal bihuler.} og svælget. Det nedre respirationssystem indeholder larynx, trachea, bronkier, bronkioler og lungealveoler. \citep{Martini2012, Sand2008}

\subsubsection{Transport af ilt} 
Ilten indtages i det øvre respirationssystem gennem næsen eller munden, herefter transporteres luften gennem trachea, hvor den deler sig i to hovedbronkier. Hovedbronkierne deler sig igen i mindre og mindre bronkier, hvor de til sidst ender i alveoler. Disse har udposninger yderst, kaldet bronkioler, og er omgivet af lungekapillærer.  Alveolevæggen er tynd, hvilket medvirker til at ilten kan diffundere over i blodet gennem lungekapillærer, hvorved blodet iltes og kan transporteres ud i musklerne. Modsat kan affaldsstoffet, kuldioxid, trænge fra blodet over i alveolerne, hvorved det kan udskilles ved eksspiration.\citep{Martini2012, Sand2008}
